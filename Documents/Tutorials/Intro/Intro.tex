\documentclass{beamer}
\usepackage{epsfig}
\title{An introduction to the AHIR-v2 tool-set}
\author{Madhav P. Desai}
\begin{document}
\maketitle



\frame[containsverbatim]{\frametitle{The purpose of the toolset}

\begin{itemize}
\item Aid in the design and implementation of complex systems (in hardware).
\item Describe the algorithm (behaviour) of the system using high-level
languages (currently C), and migrate the algorithm to a hardware
implementation.
\end{itemize}
}


\frame[containsverbatim]{\frametitle{For example}
Design a circuit that takes two inputs, computes the maximum of
the two and returns the maximum.
\begin{verbatim}
uint32_t maxOfTwo(uint32_t a, uint32_t b)
{
  uint32_t c = ((a > b) ? a : b);
}
\end{verbatim}
}


\frame[containsverbatim]{\frametitle{Lets write a test-program to check this code}
\begin{verbatim}
uint32_t maxOfTwo(uint32_t, uint32_t);
int main(int argc, char* argv[])
{
  if(argc < 3)
  {
     printf(stderr, "%s <uint32_t> <uint32_t>\n", argv[0]);
     return(1);
  }
  uint32_t c = maxOfTwo(atoi(argv[1]), atoi(argv[2]));
  printf(stdout,"Result = %d.\n", c);
  return(0);
}
\end{verbatim}
}

\frame[containsverbatim]{\frametitle{We compile and run it}

Go to ex1/ and check it out.

}


\frame[containsverbatim]{\frametitle{Conversion to hardware}

The AHIR-v2 flow translates the code to hardware in four steps:
\begin{enumerate}
\item The source C code is compiled using the open-source compiler
{\bf clang} (we version 2.8), which produces byte-code in {\bf llvm}
format.
\item We take the {\bf llvm} bytecode and translate it to our internal
representation called {\bf Aa}.  This is done by a utility called {\bf llvm2aa}.
\item Multiple {\bf Aa} files can be linked using an {\bf Aa} linker
tool called {\bf AaLinkExtMem}.
\item The linked {\bf Aa} code is then converted to a virtual circuit using
a {\bf vC} representation.  This is done using a utility called {\bf aa2vc}.
\item Finally, the {\bf vC} code is converted to VHDL which can then
be simulated or synthesized.  This conversion is done using a utility
called {\bf vc2vhdl}.
\end{enumerate}
At each stage, various optimizations are carried out.
}

\frame[containsverbatim]{\frametitle{Lets see what happens}

Go back to ex1/ and check it out.

}


\frame[containsverbatim]{\frametitle{Validating the VHDL}

\begin{itemize}
\item In order to test the VHDL, we will use the same program
that was used to test the original C code.
\item Two processes are started: one executing the test-program
and the other executing a VHDL simulator which simulates the
generated VHDL.  The two processes communicate through sockets.
\item There is no need to write a new test-bench!
\end{itemize}
}


\frame[containsverbatim]{\frametitle{Lets take a look}
Go back to examples/ex1 and check it out.
}

\frame[containsverbatim]{\frametitle{Lets make it a  little more interesting}
\begin{verbatim}
void maxDaemon()
{
  while(1)
  {
       uint32_t a = read_uint32("in_data");
       uint32_t b = read_uint32("in_data");
       c = ((a > b) ? a : b);
       write_uint32("out_data", c);
  }
}
\end{verbatim}
}

\frame[containsverbatim]{\frametitle{The testbench for the Daemon}
\begin{verbatim}
int main(int argc, char* argv[])
{
  uint32_t a, b, c;
  while(1)
  {
     scanf("%d", &a);
     scanf("%d", &b);
     write_uint32("in_data", a);
     write_uint32("in_data", b);
     c = read_uint32("out_data");
     printf(stdout,"Result = %d.\n", c);
  }
  return(0);
}
\end{verbatim}
}


\frame[containsverbatim]{\frametitle{The testbench for the Daemon}
Lets check out the software and the hardware generated from it. 
Go to ex2/.
}



\frame[containsverbatim]{\frametitle{Installing AHIR-v2}

\begin{itemize}
\item You will need to install llvm-2.8, clang-2.8, the BOOST graph library.
\item You can download the source code from gitHub.
\item Go to the v2/ folder in the git repository.  You will need to
source ``build\_bashrc'', and then type ``scons'' to build the executables.
You can then make a release by going back to the root of the git repository
and then typing ``make -f ReleaseMakeFile''.  This populates the Release
directory with binaries, examples etc. which should be enough to get you
going.
\end{itemize}
}

\end{document}
