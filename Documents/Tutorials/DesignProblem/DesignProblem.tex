\documentclass{beamer}
\usepackage{graphics}
\title{An Exercise}
\author{Madhav P. Desai\\ Department of Electrical Engg.\\ IIT Bombay, Mumbai India}
\date{March 8, 2018}
\begin{document}
\maketitle



\frame[containsverbatim]{\frametitle{Specification}

\begin{itemize}
\item Design an 8-tap programmable FIR filter (PFIRF).
\item The PFIRF has a single 32-bit FIFO input and a single
32-bit FIFO output.
\item After reset, the PFIRF first reads the 8 tap values from the FIFO input,
and on subsequent inputs, starts functioning like an FIR.
\end{itemize}
}

\frame[containsverbatim]{\frametitle{Possible solution}
\begin{verbatim}
void pfirf ()
{
    int i;
    float taps[8];
    float xvals[8];
    for (i = 0; i < 8; i++)
    {
       taps[i] = read_float32("in_data");
       xvals [i] = 0;
    }
\end{verbatim}
}

\frame[containsverbatim]{\frametitle{Possible solution: continued}
\begin{verbatim}
    int HEAD = 7;
    while (1) // outer loop.
    {
      // read input sample x and store it.
      float x = read_float32("in_data");
      xvals [HEAD] =  x;  
      // calculate the FIR output
      float output_val = 0.0;
      // inner loop..
      for (i = 0; i < 8; i++)
        output_val += xvals [(i + HEAD) & 0x7]*taps[7 -i]; 
      // Send FIR output on pipe out_data.
      write_float32 ("out_data", output_val);
      // Adjust head of circular queue.
      HEAD = (HEAD - 1) & 0x7;
    }
}
\end{verbatim}
}

\frame[containsverbatim]{\frametitle{Try to write a test-bench}
\begin{itemize}
\item Testbench should first send in the taps.
\item Then it should send in the filter inputs and
read out the filter outputs.
\item Try to do this in a burst so that you can check filter 
throughput.
\end{itemize}
}

\frame[containsverbatim]{\frametitle{Makefile}
See the model in vanillaC/
}

\frame[containsverbatim]{\frametitle{Can we do better?}
\begin{itemize}
\item Can you optimize the performance?  
\begin{itemize}
\item Use loop pipelining.
\item Use loop unrolling.
\end{itemize}
\end{itemize}
}

\frame[containsverbatim]{\frametitle{How about writing the PFIRF in {\bf Aa}}
\begin{itemize}
\item Almost exactly like C, but use a branch block and do-while.
\item Inner loop is unrolled 4-times and pipelined.
\item 1.5X better performance with marginally more hardware.
\end{itemize}
}

\frame[containsverbatim]{\frametitle{Makefile for Aa}

See the model in optAa/.   We will convert the Aa to C to 
verify the functionality before going to hardware.

}

\frame[containsverbatim]{\frametitle{Other interesting problems}
\begin{itemize}
\item Sorter:  read N integers, sort them and output the sorted list.
\item Correlator: Given an input sequence and a reference sequence, find
the amount by which the input sequence needs to be rotated to maximize
its correlation with the reference sequence. 
\item Etc..
\end{itemize}
}

\end{document}
