\documentclass{beamer}
\usepackage{graphics}
\title{Optimizations carried out by the AHIR tool chain}
\author{Madhav P. Desai\\Department of Electrical Engineering\\ IIT Bombay, Mumbai India}
\date{March 8, 2018}
\begin{document}
\maketitle


\frame[containsverbatim]{\frametitle{Optimizations}
\begin{itemize}
\item Dependency analysis.
\begin{itemize}
\item Reduce latency through a group of statements.
\end{itemize}
\item Loop pipelining.
\begin{itemize}
\item Increase throughput of a loop.
\end{itemize}
\item Operator sharing.
\begin{itemize}
\item Reduce cost of hardware.
\end{itemize}
\item Identification of disjoint memory spaces.
\begin{itemize}
\item Increase memory bandwidth.
\end{itemize}
\end{itemize}
}

\frame[containsverbatim]{\frametitle{Dependency analysis}
\begin{verbatim}
   a := (b + c)
   d := (b + c)
   e := (a + d)
\end{verbatim}
The first two statements can be executed concurrently, but the third depends
on the first two (RAW dependency).  Other dependencies that are analyzed and
enforced are WAR and WAW.  For pipe accesses, RAR dependencies are also imposed.
}

\frame[containsverbatim]{\frametitle{Loop pipelining}
A loop may be implemented as a pipeline in hardware.  We will discuss this
separately.
}

\frame[containsverbatim]{\frametitle{Operator sharing}

\begin{verbatim}
   a := (b + c)
   d := (b + c)
   e := (a + d)
\end{verbatim}
If $+$ can be replicated, we observe that only two copies of $+$ are required
to implement the three additions.

}

\frame[containsverbatim]{\frametitle{Storage Variables and Memory Spaces}
\begin{itemize}
\item Storage variables are grouped into memory spaces.
\item Each memory space is implemented in hardware by a memory sub-system. 
\item Memory spaces are identified by doing a static aliasing analysis.
\end{itemize}
}


\frame[containsverbatim]{\frametitle{Aliasing analysis}
\begin{verbatim}
int A, B, C, D; // storage objects.
int foo()
{
   int ret_val = 0;
   int* p1 = &A;
   ret_val += *p1;
   int *p2 = &B;
   ret_val += *p2;
   int *p3 = &C;
   ret_val += *p3;
   p3 = &D;
   ret_val += *p3;
   return(ret_val);
}
\end{verbatim}
}


\frame[containsverbatim]{\frametitle{Groupings}

\begin{itemize}
\item The memory spaces are $\{A\}$, $\{ B\}$, $\{ C, D \}$.
\item $C$ and $D$ are grouped together because pointer $p3$ can
point to either $C$ or $D$.
\end{itemize}

}

\frame[containsverbatim]{\frametitle{Note: Orphan references}

An orphan reference is pointer which points to an unknown storage object.
In such cases, the orphan pointer is directed to a default storage object
(sort of like /dev/null).

\begin{verbatim}
AaLinkExtMem -I 1 -E mempool prog1.aa prog2.aa 
       | vcFormat > linked.aa
Aa2VC -E mempool  linked.aa ....
\end{verbatim}

All orphan pointers in linked.aa now point to mempool which is a dummy storage object whose
size is $1$ byte.  This mempool should never actually be accessed...

}

\end{document}
