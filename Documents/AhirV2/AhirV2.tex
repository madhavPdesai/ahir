\documentclass{article}

\title{AhirV2: from algorithms to hardware}
\author{Madhav Desai \\ Department of Electrical Engineering \\ Indian Institute of Technology \\
	Mumbai 400076 India}

\newcommand{\Aa}{{\bf Aa}~}
\newcommand{\vC}{{\bf vC}~}

\begin{document}
\maketitle

\section{Introduction}

We describe the essential features
of the second version of the AhirV2 toolset
developed at IIT-Bombay.

\section{What is AhirV2?}

Perhaps the simplest way to understand AhirV2 is
through an example.

Suppose we start with a simple program.
\begin{verbatim}
int add(int a, int b)
{
        int c = (a+b);
        return(c);
}
\end{verbatim}
This program describes an algorithm.  We normally
compile this program and convert it to a machine-level
program which is executed on a processor.

What if we wished to convert this program to
a circuit?  
Such a circuit would have inputs 
$a,b$ and an output $c$ (together with some
interface hand-shake signals).  When activated,
the circuit should read its inputs and compute 
a response which is the same as would be expected
if the program were executed on a computer.

AhirV2 consists of a set of tools which takes
a C/C++ program and converts this collection to
a hardware circuit (described in VHDL).   This
conversion is done in two steps:
\begin{itemize}
\item The high-level program is compiled
to an interemediate assembly form.  AhirV2
introduces an intermediate assembly language
\Aa  which can serve as a target for sequential
programming languages (such as C/C++) as
well as for parallel programming languages.
\item From the \Aa description, a virtual
circuit (described in a virtual circuit
description language \vC) is generated.
The chief optimization carried out at this
step is that decomposing the system memory
into disjoint segments based on usage
analysis (this considerably improves the
available memory bandwidth and reduces
system cost).
\item From the \vC description, a
VHDL description of the system is generated.
The chief optimization carried out at this
stage is resource sharing.  The \vC description
is analyzed to identify operations which
cannot be concurrently active and this information
is used to reduce the hardware required.
\item 
The VHDL description produced from 
\vC is in terms of a library
of VHDL design units which has been developed
as part of the AhirV2  effort.  This library
consists of control-flow elements, data-path
elements and memory elements.
\item 
For the C/C++ to \Aa translation, the
AhirV2 toolset uses the front-end C/C++ compiler
infrastructure developed by the LLVM project (www.llvm.org), which
compiles the C/C++ program to LLVM byte-code
(which is a processor independent representation
of the compiled program).  The AhirV2 toolset
has a utility which translates the LLVM byte-code
to an \Aa description.
\end{itemize}
In Section \ref{sec:Example}, 
we illustrate the use of these tools on the
simple example shown above.

The earlier version of the AHIR flow has been
used to convert non-trivial programs to hardware.
The resulting circuits are upto two orders of
magnitude more energy-efficient than
a processor \cite{ref:dsd2010}.  The AhirV2
toolset incorporates several optimizations which
make the resulting circuits more 
competitive. 

\section{The tools}

We assume that you have access to either {\bf llvm-gcc}
of {\bf clang} as the front-end compiler which generates
LLVM byte-code from C/C++.  The current AhirV2 toolset
is consistent with llvm 2.8 and clang 2.8.

The other tools in the chain are described below.

\subsection{{\bf llvm2aa}}

This tool takes LLVM byte code and converts it into an
\Aa file.
\begin{verbatim}
llvm2aa bytecode.o > bytecode.aa
\end{verbatim}
The generated \Aa code is sent to {\bf stdout} and all informational
messages are sent to {\bf stderr}.  On success, the tool returns 0.


\subsection{{\bf Aa2VC}}

This tool takes a list of \Aa programs and converts them
to a \vC description. 
\begin{verbatim}
Aa2VC [-O] file1.aa file2.aa ...  > result.vc
\end{verbatim}
The generated \vC code is sent to {\bf stdout} and all informational
messages are sent to {\bf stderr}.  On success the tool returns 0.

The tool performs memory space decomposition and if the {\bf -O} option
is specified, it also attempts to parallelize sequential code using 
dependency analysis.

\subsection{\bf vc2vhdl}

Takes a collection of \vC descriptions and converts them to
VHDL.
\begin{verbatim}
vc2vhdl -t foo [-t bar -t bar2 ...]  -f file1.vc -f file2.vc ...  > system.vhdl
\end{verbatim}
Using the {\bf -t} option, one can specify the modules which are to be 
accessible from the ports of the generated VHDL system.
Using the {\bf -f} option, one specifies the \vC files to be analyzed.

The tool performs concurrency analysis to determine operations which
can be mapped to the same physical operator without the need for
arbitration.  It also instantiates separate memory subsystems for
the disjoint memory spaces (in practice many of the memory spaces
are small and are converted to register banks).

\subsection{Miscellaneous: {\bf vcFormat} and {\bf vhdlFormat}}

The outputs produces by {\bf Aa2VC} and {\bf vc2vhdl} are
not well formatted.  One can format \Aa and \vC files
using  {\bf vcFormat} as follows
\begin{verbatim}
vcFormat < unformatted-vc/aa-file  > formatted-vc/aa-file
\end{verbatim}
and similarly use {\bf vhdlFormat} to format generated
VHDL file.


\section{An example} \label{sec:Example}

Let us revisit the simple example considered in
the first section:
\begin{verbatim}
int add(int a, int b)
{
        int c = (a+b);
        return(c);
}
\end{verbatim}
We wish to generate a circuit which {\em implements}
the specification implied by this program.

We convert the program to LLVM byte code using
the {\bf clang} compiler (www.llvm.org)
\begin{verbatim}
      clang -std=gnu89 -emit-llvm -c add.c
\end{verbatim}
This produces a binary file {\bf add.o} which is
the LLVM byte-code.  To make the byte-code human
readable, we dis-assemble it using an LLVM utility
\begin{verbatim}
     llvm-dis add.o
\end{verbatim}
This is what the LLVM assembly code looks like
\begin{verbatim}
; ModuleID = 'add.o'
target datalayout = "e-p ...... "
target triple = "i386-pc-linux-gnu"

define i32 @add(i32 %a, i32 %b) nounwind {
  %1 = alloca i32, align 4
  %2 = alloca i32, align 4
  %c = alloca i32, align 4
  store i32 %a, i32* %1, align 4
  store i32 %b, i32* %2, align 4
  %3 = load i32* %1, align 4
  %4 = load i32* %2, align 4
  %5 = add nsw i32 %3, %4
  store i32 %5, i32* %c, align 4
  %6 = load i32* %c, align 4
  ret i32 %6
}
\end{verbatim}
To get to this point, we could have used several
optimizations which are available in the LLVM frame-work.
But we work with the unoptimized version to illustrate
the storage decomposition which is carried out by
the AhirV2 tools.

The LLVM byte-code is our starting point.  We first convert it
to \Aa.
\begin{verbatim}
llvm2aa add.o | vcFormat > add.o.aa
\end{verbatim}
This produces an \Aa program
\begin{verbatim}
// Aa code produced by llvm2aa (version 1.0)
$module [add]
// arguments
$in (a : $uint<32> b : $uint<32> )
$out (ret_val__ : $uint<32>)
$is
{
  $storage stored_ret_val__ : $uint<32>
  $branchblock [add]
  {
    //begin: basic-block bb_0
    $storage iNsTr_0 : $uint<32>
    $storage iNsTr_1 : $uint<32>
    $storage c : $uint<32>
    iNsTr_0 := a
    iNsTr_1 := b
    // load
    iNsTr_4 := iNsTr_0
    // load
    iNsTr_5 := iNsTr_1
    iNsTr_6 := (iNsTr_4 + iNsTr_5)
    c := iNsTr_6
    // load
    iNsTr_8 := c
    stored_ret_val__ := iNsTr_8
    $place [return__]
    $merge return__ $endmerge
    ret_val__ := stored_ret_val__
  }
}
\end{verbatim}

Now, this \Aa code is converted to a virtual circuit \vC representation.
\begin{verbatim}
     Aa2VC -O add.o.aa | vcFormat > add.o.aa.vc
\end{verbatim}
The virtual circuit representation is a bit too verbose to reproduce entirely
here, but we show some critical fragments
\begin{verbatim}
$module [add] 
{
  $in a:$int<32> b:$int<32>
  $out ret_val__:$int<32>
  $memoryspace [memory_space_0] 
  {
    $capacity 1
    $datawidth 32
    $addrwidth 1
    // ret-val is kept here
    $object [xxaddxxstored_ret_val__] : $int<32>
  }
  $memoryspace [memory_space_1] 
  {
    $capacity 1
    $datawidth 32
    $addrwidth 1
    // a is kept here.
    $object [xxaddxxaddxxiNsTr_0] : $int<32>
  }
  $memoryspace [memory_space_2] 
  {
    $capacity 1
    $datawidth 32
    $addrwidth 1
    // b is kept her
    $object [xxaddxxaddxxiNsTr_1] : $int<32>
  }
  $memoryspace [memory_space_3] 
  {
    $capacity 1
    $datawidth 32
    $addrwidth 1
    // c is kept here.
    $object [xxaddxxaddxxc] : $int<32>
  }
  $CP 
  {
     // a control-flow petri-net..  verbose..
  }
  // end control-path
  $DP 
  {
     // wires and operators.
  }

   // links between CP and DP
}
\end{verbatim}
The important points to note are that the stored objects a,b,c and
ret\_val\_\_ are mapped to different memory spaces.  Thus, the
chief difference between a \vC description and a processor is 
that the \vC program partitions storage into small units which
are accessed only by operators that need them. 

Finally, we take the \vC description and convert it to
VHDL
\begin{verbatim}
vc2vhdl -t add -f add.o.aa.vc | vhdlFormat > system.vhdl
\end{verbatim}
This produces a VHDL implementation of the system with
{\bf add} marked as a top-level module.  The VHDL that
is produced is too voluminous to reproduce here, but
the top-level system entity is
\begin{verbatim}
entity test_system is  -- system
  port (--
    add_a : in  std_logic_vector(31 downto 0);
    add_b : in  std_logic_vector(31 downto 0);
    add_ret_val_x_x : out  std_logic_vector(31 downto 0);
    add_tag_in: in std_logic_vector(0 downto 0);
    add_tag_out: out std_logic_vector(0 downto 0);
    add_start : in std_logic;
    add_fin   : out std_logic;
    clk : in std_logic;
    reset : in std_logic); --
  --
end entity;
\end{verbatim}
There are ports corresponding to the arguments of the top-level module,
and the add\_start/add\_fin is a handshake pair.  One sets of the inputs,
starts the system and waits until the fin is asserted.  After the fin
is asserted, one has the return value of at the appropriate port.

\section{Handling complex programs}

This example is trivial.  Real, non-trivial programs can be
mapped in this manner.  Currently, there are only
two restrictions
\begin{itemize}
\item No recursion, no cycles in the call-graph of the original
program.
\item No function pointers.
\end{itemize}

The AhirV2 flow produces a modular system, and by default,
produces one VHDL entity for every function in the original
program.  Further, the concept of message pipes is native
to the \Aa and \vC descriptions.  Thus, it is easy to
map parallel programs or programs consisting of multiple
concurrent processes to hardware.

\section{To be continued}

This document is a work-in-progress.  More details and
related documentation will be added shortly.

The examples folder contains some explanations which might
make things easier to understand.

\begin{thebibliography}{99}
\bibitem{ref:dsd2010}
Sameer D. Sahasrabuddhe, Sreenivas Subramanian, Kunal P. Ghosh, Kavi Arya, Madhav P. Desai, 
"A C-to-RTL Flow as an Energy Efficient Alternative to Embedded Processors in Digital Systems," 
DSD, pp.147-154, 2010 13th Euromicro Conference on Digital 
System Design: Architectures, Methods and Tools, 2010
\end{thebibliography}
\end{document}
