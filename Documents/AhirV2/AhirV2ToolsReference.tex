%%\begin{figure}
%%\begin{centering}
%%\centerline{\psfig{figure=OurApproach.eps,width=3.0in,height=3.0in}}
%%\caption{Our Approach to Fault Simulation using a Hardware Emulator}
%% \label{fig:OurApproach}
%%\end{centering}
%%\end{figure}

\documentclass[12pt]{article}
\usepackage{epsfig}
\usepackage{graphics}

\title{AhirV2: Tools which take algorithms to hardware \\ A reference manual }
\author{Madhav Desai \\ Department of Electrical Engineering \\ Indian Institute of Technology \\
	Mumbai 400076 India}

\newcommand{\Aa}{{\bf Aa}~}
\newcommand{\vC}{{\bf vC}~}

\begin{document}
\maketitle

\section{What is AhirV2?}

AhirV2 is a set of tools which can convert a C 
description of a system to  an equivalent hardware
implementation (described in VHDL).  Using these tools,
it is possible to take an algorithmic approach to 
the design of  hardware.

The flow of transformations is illustrated in Figure \ref{fig:AhirFlow}.
\begin{figure}
\centering
\includegraphics[width=8cm]{AHIRv2Flow.pdf}
\caption{AhirV2 flow}
 \label{fig:AhirFlow}
\end{figure}

\begin{itemize}
\item Given a high-level C, we rely on
an LLVM (www.llvm.org) compatible compiler such as clang (www.clang.org)
to produce LLVM byte code.
Currently, the AhirV2 flow uses LLVM byte code as a starting point.
\item The LLVM byte-code program is compiled
to an intermediate assembly form.  AhirV2
introduces an intermediate assembly language
\Aa  which serves as a target for sequential
programming languages (such as C) as
well as for parallel programming languages.
An \Aa program consists of modules (analogous
to sub-programs in C) which can call each 
other, and can communicate through storage objects
as well as through pipes (first-in-first-out buffers).
\item From the \Aa description, a virtual
circuit (described in a virtual circuit
description language \vC) is generated.
The chief optimizations carried out at this
step are dependency based operation
ordering, dynamic loop-pipelining and decomposition of the system memory
into disjoint spaces based on static pointer
analysis (this considerably improves the
available memory bandwidth and reduces
system cost).  A \vC description also
consists of modules: however, the modules
are presented in a factored form (control X data X storage).
\item From the \vC description, a
VHDL description of the system is generated.
The system consists of modules, memory spaces
and FIFO buffers.  The modules are further
broken down into a control-path (a live and safe
Petri net), and a data-path (a graph of operators
and wires).
The chief optimization carried out at this
stage is resource sharing.  The \vC description
is analyzed to identify operations which
cannot be concurrently active and this information
is used to reduce the hardware required.
\item 
The VHDL description produced from 
\vC is in terms of a library
of VHDL design units which has been developed
as part of the AhirV2  effort.  This library
consists of control-flow elements, data-path
elements and memory elements.
\end{itemize}

Thus, to generate hardware using the AhirV2 flow,
it is possible to start at the C-level, at the \Aa level
at the \vC level or at the VHDL level (or a combination
of all these levels).  Starting at a higher level is
easier for the programmer, but using lower level representations
will usually lead to more efficient hardware.

Currently, there are only two restrictions in mapping
a C program to VHDL using the AhirV2 flow:
\begin{itemize}
\item No recursion, no cycles in the call-graph of the original
program.
\item No function pointers.
\end{itemize}

\section{The tools}

We assume that you have access to either {\bf llvm-gcc}
or {\bf clang} as the front-end compiler which generates
LLVM byte-code from C/C++.  The current AhirV2 toolset
is consistent with llvm 2.8 and clang 2.8.

The other tools in the chain are described below.

\subsection{{\bf llvm2aa}}

This tool takes LLVM byte code and converts it into an
\Aa file.
\begin{verbatim}
llvm2aa options bytecode.o > bytecode.aa
\end{verbatim}
The generated \Aa code is sent to {\bf stdout} and all informational
messages are sent to {\bf stderr}.  On success, the tool returns 0.

The options:
\begin{itemize}
\item {\bf -modules=listfile} : Specify the list of functions in the bytecode
which should be converted to \Aa.   The names of these functions should be
listed in the text-file listfile. If absent, all functions
are converted.
\item {\bf -storageinit} :  Storage objects in the llvm bytecode
are explicitly initialized in the generated \Aa code.   An initializer
routine named
\begin{verbatim}
    global_storage_initializer
\end{verbatim}
is instantiated in
the \Aa code for this purpose.
\item {\bf -pipedepths=filename} : Specifies a file which contains
the depths of pipes which are part of the generated \Aa code.
\item {\bf -extract\_do\_while} : Innermost loops which are marked
using a call to the special function 
\begin{verbatim}
    _loop_pipelining_on_
\end{verbatim}
are extracted as pipelined do-while loops.  This is necessary
for automatic cross-iteration parallelization of inner loops in the 
generated hardware (substantial performance benefits can be
realized).
\item There are zillions of other LLVM optimizations that are
available in llvm2aa (type llvm2aa --help to see these).  These
involve program level optimizations such as loop unrolling,
garbage collection optimizations etc. In general anything that
increases instruction level parallelism will result in faster
hardware, but with a cost.
\end{itemize}

\subsection{{\bf AaLinkExtMem}}

This linker tool takes a list of \Aa files, elaborates the program,
creates a global storage initializer, and
does memory space decomposition.  The externally visible memory space is
linked in one of two ways: either it is assumed to be external
and all accesses to it are routed out of the \Aa program,
or it is assumed to be internal and assumed to consist of
a memory object (an array of bytes).  External pointer dereferences
are handled as if they are directed at this memory object.
\begin{verbatim}
AaLinkExtMem options file1.aa file2.aa ...  > linked.aa
\end{verbatim}
The generated \Aa code is sent to {\bf stdout} and all informational
messages are sent to {\bf stderr}.  On success, the tool returns 0.

The options:
\begin{itemize}
\item {\bf -I n}: specifies that external references to memory
are to be mapped as if they are to an internal object whose size
is $n$ bytes.
\item {\bf -E obj-name} : specifies that the object to which
external references are mapped is to be named obj-name.
\end{itemize}
We recommend that you use the {\bf -I} and {\bf -E} options to
locate externally visible memory into a specified object in the
\Aa program.   

If the {\bf -I} option is not used, then all external memory
references are routed out of the \Aa program through pipes.
In this case, if the \Aa compiler determines that there is some pointer 
in the program which can point
to both internal and external memory, then this will be
declared as an error!  

If the programs being linked contain memory initialization
routines, the linker generates a global storage initialization
function which is named 
\begin{verbatim}
global_storage_initializer
\end{verbatim}
This global initializer calls all the memory initializers in
the programs being linked.

\subsection{{\bf AaOpt}}

The optimization utility {\bf AaOpt} takes an \Aa program (list of
\Aa files) and produces an optimized version of the source program.
\begin{verbatim}
AaOpt options file1.aa file2.aa ... > optimized.aa
\end{verbatim}
The optimized \Aa code is printed to {\bf stdout}.  On success, 
the tool returns a 0 (else a non-zero).  Macro and inlined
function calls in the source code are substituted in place
in the optimized code.

The options:
\begin{itemize}
\item {\bf -r module-name} (optional) : specifies a root module in the
system.  Multiple root modules can be specified.  All dead code (which
is not reachable from a root module) is eliminated.
\item {\bf -I extmem-object} (optional) : similar to AaLinkExtMem,
this option specifies the name of the extmem-object in the
source \Aa files.
\item {-C} (optional) : if specified, try to eliminate registers
to the maximum extent possible by marking statements which 
have a fanout of one as volatile.
\item {-B} (optional) : if specified, add buffering to balance pipelined
loops so that loop performance is not bottlenecked by inadequate
buffering. 
\end{itemize}
Note that the -B and -C options cannot be used together.  


\subsection{{\bf Aa2VC}}

This tool takes a list of \Aa programs and converts them
to a \vC description. 
\begin{verbatim}
Aa2VC options file1.aa file2.aa ...  > result.vc
\end{verbatim}
The generated \vC code is sent to {\bf stdout} and all informational
messages are sent to {\bf stderr}.  On success the tool returns 0.

The options:
\begin{itemize}
\item {\bf -h}: print help message and quit.
\item {\bf -O} : if used, sequential statement blocks are parallelized
by doing dependency analysis.
\item {\bf -C} : if used, a C stub is created for every module that
is not called from within the system.  These stubs can be used to
interface to a VHDL simulator (or even drive hardware) to verify
the VHDL code generated by downstream tools.
\item {\bf -U} : memory subsystems will be unordered (that is,
will not guarantee in-order completion of accesses).  This
leads to a simpler memory subsystem, but more conservative
control flow.  The default is that all memory subsystems
are ordered (will complete read/write requests in the order that
they are accepted).
\item {\bf -r root-module} (optional): specifies a root module.
Code which is not accessible from a root-module is considered
as dead code and is ignored.
\item {\bf -I obj-name} : if specified, all external memory references
are considered as being directed at the storage object named obj-name.
If not specified, then the tool will throw an error if it finds
a pointer dereference that cannot be resolved as pointing only to
storage objects declared inside the \Aa program.
\item {\bf -P} (optional):  Normally, the {\bf Aa} compiler resolves
accesses to pipes and storage objects by analysing modules.  But if
the -P flag is specified, all modules are treated as opaque and
are assumed to not have any side-effects.  This can be dangerous.
\end{itemize}

\subsection{\bf vc2vhdl}

Takes a collection of \vC descriptions and converts them to
an AHIR system described in VHDL.
\begin{verbatim}
vc2vhdl [-O] [-C] [-q] [-a] [-e <entity-name] [-w]
         -t/-T foo [-t/-T bar -t/-T bar2 ...]
         [-s ghdl/modelsim] [-L <file-name>]
         [-v] [-D] [-W <work-lib>]
         [-U] [-H]
         -f file1.vc -f file2.vc ...  > system.vhdl
\end{verbatim}

The options:
\begin{itemize}
\item {\bf -t} : to specify the modules which are to be 
accessible from the ports of the generated VHDL system.
Such modules have to be top-level (that is, they cannot
be called from within the program).
Multiple top-level modules can be specified in this way.
The control and argument ports for these modules are
visible at the interface of the generated AHIR system.
\item {\bf -T} : to specify top-level modules which are to be 
free-running inside the AHIR system.
Multiple top-level modules can be specified in this way.
Such modules do not have any arguments and do not return
any values.  Their only mechanism of communication with
the world outside the AHIR system is through pipes.
The control ports for these modules are
{\bf not visible} at the interface of the generated AHIR system.
In the AHIR system, these modules are started on reset
and are run forever (restarted after they finish, forever).
\item {\bf -f file-name} : specifies the \vC files to be analyzed. 
Multiple \vC files may be specified.  An object must be
defined before it is used, so the \vC files must be 
specified in the correct order.
\item {\bf -O} : optimize the generated VHDL by compacting
the control-path.  This does not change the resulting
hardware, but makes the generated VHDL file smaller.
\item {\bf -C} : the VHDL code has a system test bench which
interfaces to foreign code using a VHPI/Modelsim-FLI interface.
If this is not specified, the  generated test bench simply
instantiates the system and starts all top-level modules
off (you will need to fill in your own test bench here).
The C testbench is usually easier to write (it probably
already exists in the form of the original program).
\item {\bf -a} : try to minimize the area of the resulting
VHDL by sharing operators to the maximum extent possible
(allowing potential contention for resources).  This will
result in a slower (usually by 2X) system, but will
also reduce the area (usually by 0.5X).  If not specified,
two operations will be mapped to the same
operator  only if it can be proved that they cannot be active simultaneously.
\item {\bf -q } : if specified, do aggressive register insertion
to minimize the clock period.
\item {\bf -S bypass-stride } : by specifying the bypass stride
(an integer $\geq 1$),
the user can trade-off clock cycles versus clock period.  
The lowest clock period will be obtained for -S 1.   
\item {\bf -e top-entity-name} : The generated top-level VHDL entity
corresponding to the AHIR system is named top-entity-name.  The default
is ahir\_system.
\item {-L function-library} : AhirV2 provides some built in operator
functions which can be called from your code.  These are organized
as function libraries and this option specifies a function library
to look into when generating VHDL.  For example {\em -L fpu} gives
access to the floating point library which provides some useful
built in functions (e.g. fpalu32, fpalu64 etc.).
\item {\bf -w} :  If specified, the VHDL system and test-bench are
generated as separate unformatted VHDL files.  You will need to
format these using the vhdlFormat command.
\item {\bf -s ghdl/modelsim} :  If {\bf ghdl} is specified
with the -s option, then the generated testbench (if -C is specified)
uses the VHPI interface to link with foreign code.  Otherwise,
the generated testbench (if -C is specified) uses the Modelsim FLI
interface to link with foreign code.
\item {\bf -U}: input/output pipes will be printed with depth 0. 
\item {\bf -H}: system interface in .hsys format will be printed.   This
is used by another AHIR-V2 tool called hierSysBuild which can put together
multiple AHIR systems using this hsys information.
\end{itemize}

The tool performs concurrency analysis to determine operations which
can be mapped to the same physical operator without the need for
arbitration.  It also instantiates separate memory subsystems for
the disjoint memory spaces (in practice many of the memory spaces
are small and are converted to register banks).


\subsection{{\bf Aa2C}: convert an {\bf Aa} description into a {\bf C} program}

The AhirV2 flow offers considerable flexibility to a system designer.
For example, it is possible to write code directly in {\bf Aa} in order
to get more optimal implementations (relative to those obtained 
starting from {\bf C}).  In such cases, if we wish to simulate
the {\bf Aa} description, we would use the {\bf Aa2C} utility to
convert the {\bf Aa} code to ANSI {\bf C}, and then compile it
in the usual way.

The {\bf Aa2C} program can be summarized as
\begin{verbatim}
Aa2C [-I <ext-mem-object>] [-P <prefix>]  <aa-file> (<aa-file>)*
\end{verbatim}
The only option is:
\begin{itemize}
\item {\bf -I ext-mem-object}:  the same behaviour as in {\bf Aa2VC}.
\item {\bf -P prefix}:  Specify a prefix string (default is the empty string)
which will be used to name generated files (see below).
\end{itemize}
The remaining arguments are {\bf Aa} files which will be linked and
converted to {\bf C} code.  Two outputs files are created:
\begin{itemize}
\item [prefix]aa\_c\_model.h :  a header file declaring functions in the generated
source code.
\item [prefix]aa\_c\_model\_internal.h :  an internal header file with lots of macros.
\item [prefix]aa\_c\_model.c : a source file containing function definitions
corresponding to the {\bf Aa} modules.
\end{itemize}

External calls into the generated {\bf C} code must have the
form:
\begin{verbatim}
void foo ( Ctype_1 in_1, Ctype_2 in_2, Ctype_3* out_1, Ctype_4* out_2);
\end{verbatim}
where Ctype is either a float or double or (int/uint)(64/32/16/8)\_t type.
You can then link your external code with the generated {\bf C} code
in the usual way.

\subsection{{\bf AaPreprocess}: Preprocessor for {\bf Aa} description}

The {\bf AaPreprocess} utility takes an input \Aa file and applies pre-processing
directives in the file  to produce an output \Aa file.   The invocation of the
{\bf AaPreprocess} utility is as follows:
\begin{verbatim}
AaPreprocess [-I <search-directory>]* -o <output-aa-file> <list-of-input-aa-files>
\end{verbatim}

The utility then applies the pre-processing directives in the input \Aa files
to produce an output \Aa file.

The pre-process directives that are supported are
\begin{description}
\item[\#include]  Specify a file to be included.  This is typically specified
as 
\begin{verbatim}
#include foo.aa
\end{verbatim}
The directories specified by the -I option are searched and the first
matching file with name ``foo.aa'' is substituted in place at the point
of the include.
\item[\#define]  Specifies a key-word definition.  For example
\begin{verbatim}
#define P  const_56
\end{verbatim}
This defines the  string "P" to be an alias of the string "const\_56".
A \#define can override another \#define.  The last definition is used.
\item[\#undefine]  This removes the specified define from the list
of preprocessor defines.
\begin{verbatim}
#undefine P
\end{verbatim}
Any further references to ``P'' in the source will be
flagged as errors by the pre-preocessor.
\item[\#\#]  This the paste directive.   For example, if we have
defined
\begin{verbatim}
#define P  const_56
\end{verbatim}
and we have a line in the input file of the form
\begin{verbatim}
a := ( ##P + 1)
\end{verbatim}
then, this will be expanded to
\begin{verbatim}
a := (const_56 + 1)
\end{verbatim}
\item [\#if - \#endif] 
Code can be conditionally included using this construct.
\begin{verbatim}
#define TFLAG 1
#define SFLAG 0
#if TFLAG
a := b
#endif 
#if SFLAG
b := a
#endif
\end{verbatim}
will be expanded to
\begin{verbatim}
a := b
\end{verbatim}
\end{description}

To summarize: the preprocessor can be used to 
\begin{itemize}
\item include a file into the code.
\item paste strings into the code.
\item conditionally include code.
\end{itemize}
This adds flexibility to the programmer and
can reduce the amount of repetitive code.


\subsubsection{Restrictions in using {\bf Aa2C}}

The current implementation of {\bf Aa2C} produces un-threaded code.
Thus, if you have a parallel block in your {\bf Aa} code, the
statements in the parallel block are serialized in the resulting
{\bf C} program.  This can result in the generated {\bf C} program
potentially hanging (if one of the statements in the parallel
block runs for-ever).   Another situation is when two
concurrent blocks in the {\bf Aa} program are writing and reading
from the same pipe.  In such a case, the serialized code may
get dead-locked.  You need to be careful that your {\bf Aa} code
does not have such situations (the simplest option is
to not use parallel blocks in the {\bf Aa} code!).

This issue will be fixed in a future release of {\bf Aa2C}.

\subsection{Miscellaneous: {\bf vcFormat} and {\bf vhdlFormat}}

The outputs produced by {\bf Aa2VC} and {\bf vc2vhdl} are
not well formatted.  One can format \Aa and \vC files
using  {\bf vcFormat} as follows
\begin{verbatim}
vcFormat < unformatted-vc/aa-file  > formatted-vc/aa-file
\end{verbatim}
and similarly use {\bf vhdlFormat} to format generated
VHDL files.

\end{document}
