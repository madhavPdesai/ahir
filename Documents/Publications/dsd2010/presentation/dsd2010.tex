\documentclass{beamer}
\usepackage{epsfig}
\title{A C-to-RTL flow as an Alternative to the Use of Embedded Processors in Digital Systems}
\author{Sameer D. Sahasrabuddhe, Sreenivas
  Subramanian, Kunal P. Ghosh, Kavi Arya \\ Madhav P. Desai\\
  Indian Institute of Technology -- Bombay, Powai, Mumbai -- 400076, INDIA}
\begin{document}

\maketitle

\frame[containsverbatim]{\frametitle{Overview}

\begin{itemize}
\item We present a correct-by-construction and scalable 
C-to-RTL design flow. 
	\begin{itemize}
	\item with a factored intermediate representation (AHIR), in which
		control-flow, data-flow and storage are represented as concurrent
		interacting subsystems.
	\end{itemize}
\item Experimental data to characterize the C-to-RTL flow
	\begin{itemize}
	\item comparison in terms of area, delay and {\bf energy}.
	\item for a set of programs, use standard commercial synthesis tools to obtain
		 an ASIC starting from the RTL generated by the C-to-RTL flow.
	\item compare with processor implementations of the same set of programs.
	\end{itemize}
\end{itemize}

}

\frame[containsverbatim]{\frametitle{A system description in AHIR}

\begin{figure}
\begin{centering}
\centerline{\psfig{figure=AhirSystem.eps,width=2.5in,height=2.0in}}
\end{centering}
\end{figure}

A system description consists of a collection of modules (a module is
similar to a function in a C program).  The environment is responsible
for memory initialization and for providing input arguments to
the system.
}

\frame[containsverbatim]{\frametitle{An AHIR module}

An AHIR module description is of the form: 

\begin{displaymath}
{\bf CP} \ \times \ {\bf DP} \ \times \ {\bf S}
\end{displaymath}

where ${\bf CP}$ represents control flow, ${\bf DP}$ represents 
data flow, and {\bf S} represents storage.
}

\frame[containsverbatim]{\frametitle{The control path ${\bf CP}$}

\begin{itemize}
\item The control path is modeled by a petri-net.  Each transition 
is associated with a symbol, which can either be an input symbol
or an output symbol.
\item When a transition labeled by an input symbol is enabled, 
it fires when the input symbol is received.
\item When a transition labeled by an output symbol is enabled,
it fires eventually and emits the output symbol.
\item The petri-net has a special structure, which guarantees that
{\bf CP} is live and safe.
\end{itemize}
}


\frame[containsverbatim]{\frametitle{The data path ${\bf DP}$}

\begin{itemize}
\item A directed graph of operators.
\item Each operator has a set of request symbols and a set of acknowledge symbols.
The operator responds to the arrival of request symbols by eventually generating
appropriate acknowledge symbols.
\item Operator library: all the standard arithmetic and logical operators, as
well as a multiplexor and load/store operators.
\end{itemize}
}

\frame[containsverbatim]{\frametitle{The storage subsystem ${\bf S}$}
\begin{itemize}
\item The storage subsystem connects to the load/store operators in the datapath,
and is required to eventually service the load/store requests.
\end{itemize}
}

\frame[containsverbatim]{\frametitle{Illustration of an AHIR module} 
\begin{figure}
\begin{centering}
\centerline{\psfig{figure=Illustration.eps,width=2.5in,height=2.0in}}
\end{centering}
The computation being performed is
\begin{verbatim}
               x = y + 1 || w = mem(y)
               y = x + w 
\end{verbatim}
\end{figure}
}

\frame[containsverbatim]{\frametitle{Type-2 petri-nets} 
Type-2 petri-nets are live and safe petri-nets constructed using a
set of simple rules.  The simplest type-2 petri-net has the form
shown in the figure.
\begin{figure}
\begin{centering}
\centerline{\psfig{figure=Type2Base.eps,width=2.5in,height=2.0in}}
\end{centering}
\end{figure}
}


\frame[containsverbatim]{\frametitle{Production rules} 
Replace a transition by a transition-place-transition (any region produced
from a transition is called a t-region):
\begin{figure}
\begin{centering}
\centerline{\psfig{figure=Type2T1.eps,width=2.5in,height=1.5in}}
\end{centering}
\end{figure}
}

\frame[containsverbatim]{\frametitle{Production rules} 
Replace a transition by an acyclic graph constructed using
transitions and p-regions: the acyclic graph must have a
single source and a single sink.
\begin{figure}
\begin{centering}
\centerline{\psfig{figure=Type2T2.eps,width=2.5in,height=2.0in}}
\end{centering}
\end{figure}
}

\frame[containsverbatim]{\frametitle{Production rules} 
Replace a place by a possibly cyclic graph consisting of
places and t-regions.
\begin{figure}
\begin{centering}
\centerline{\psfig{figure=Type2P1.eps,width=2.3in,height=2.0in}}
\end{centering}
\end{figure}
}

\frame[containsverbatim]{\frametitle{Production rules} 
A place with more than one incoming arc may be replaced
by a parallel merge region as shown.  The region obtained by
this replacement is not allowed to be refined any further..
\begin{figure}
\begin{centering}
\centerline{\psfig{figure=Type2P2.eps,width=2.5in,height=2.0in}}
\end{centering}
\end{figure}
}


\frame[containsverbatim]{\frametitle{Liveness and Safety} 

\begin{itemize}
\item   Assume that an enable input transition eventually fires.  Then the type-2
petri-net is live and safe.
\item The type-2 petri-net is also powerful enough to describe the control
flow in imperative languages such as C/C++ as well as in synchronous
languages such as Esterel.
\end{itemize}

}

\frame[containsverbatim]{\frametitle{Optimizations} 

\begin{itemize}
\item Define two transitions to be compatible if it is not possible
for both of them to fire simultaneously.
\item The compatibility of two transitions in a type-2 petri-net can 
be determined in linear time.
\item Thus, in an AHIR module, it is easy to determine which operators
can be shared without the need for arbitration. This leads to 
considerable hardware reduction without compromising performance.
\end{itemize}
}

\frame[containsverbatim]{\frametitle{Our C-to-VHDL flow} 
\begin{itemize}
\item The starting point is a program written in C
\begin{itemize}
\item restrictions: no cycles in call graph, and no {\bf function} pointers.
\end{itemize}
\item Use the LLVM front end to convert C program into LLVM byte code (CDFG).
\item Map the CDFG to AHIR (the resulting AHIR description can be proved
to be equivalent to the CDFG).
\item Map AHIR to VHDL
\begin{itemize}
\item identify arbiter-less resource sharing opportunities to reduce hardware
cost.
\item instantiate the system: modules with their control and data-paths, the
inter-module link layer to handle calls, the memory subsystem.
\end{itemize}
\end{itemize}
}

\frame[containsverbatim]{\frametitle{Mapping AHIR to VHDL} 
\begin{itemize}
\item A synchronous, single clock, positive edge-triggered paradigm 
is used.
\item Transitions are coded as pulses which are sampled high by one
clock edge.  An output transition is implemented by an AND gate whose
inputs are the input places to the transition.
\item A place is modeled by a flip-flop which is set when any of its
input transitions fires, and is reset when any of its successor transitions
fire.
\begin{itemize}
\item A place with only a single input transition and a single output transition
is optimized away.
\end{itemize}
\item The datapath is mapped to an equivalent VHDL netlist constructed
using a library of operators.  A set of compatible operators is replaced
by a single operator with input multiplexors and output demultiplexors.
\item The memory subsystem is implemented using multiple banks, and offers
as many ports as there are shared load/store operators which use it.
\end{itemize}
}


\frame[containsverbatim]{\frametitle{Experimental Evaluation}
\begin{itemize}
\item Select a range of programs
\begin{itemize}
\item A5 (stream cipher), AES encryption, Red-black Trees (data-structure),
      Linpack (LU factorization), Fast-Fourier Transform (FFT).
\end{itemize}
\item Using the C-to-RTL flow, map each program to RTL.  The run-time for
mapping is neglible in all cases (less than a minute).
\item Use standard synthesis tools (Synopsys Design Compiler, Cadence SOC encounter)
to implement ASIC from RTL (we use the 180nm TSMC CMOS process, with OSU standard
cell libraries).  Extract area, delay and energy numbers for the implemented ASIC.
\begin{itemize}
\item Operators are not pipelined or optimized in any fashion.
\end{itemize}
\item Run each program on a  processor based platform (we use the Intel Atom N270 as
a reference).  Extract area, delay, energy numbers using the processor data sheet.
\item Compare the ASIC numbers with the processor numbers.
\end{itemize}
} 


\frame[containsverbatim]{\frametitle{Data for 180nm implementations of C-to-RTL results }

\begin{table}[htb]
  \centering
  \caption{Area/Delay/Power/Energy data for AHIR circuits in the TSMC $0.18\mu$ technology}
  \label{table:tsmc-180nm-data}
  \renewcommand\arraystretch{1.2}
  \setlength{\tabcolsep}{1ex}
  \begin{tabular}{c|c|c|c|c|c}
    \hline
    & Area & Freq & Delay & Power & Energy \\
    & (mm$^2$) & (MHz) & (ms) & (mW) & ($\mu$J) \\
    \hline
    \hline
    A5/1 & 1.45 & 71.4 & 0.28$\mu$s & 73 & 20.44\,nJ \\
    \hline
    AES & 6.5 & 71.4 & 0.428 & 338 & 144.7 \\
    \hline
    FFT & 5.1 & 41.7 & 0.155  & 95 & 14.7  \\
    \hline
    LPK & 27 & 41.7 & 37.7 & 273 & 10300  \\
    \hline
    RBT & 18 & 41.7 & 9.9 & 153 & 1511 \\
    \hline
  \end{tabular}
\end{table}

}


\frame[containsverbatim]{\frametitle{Data for 45nm Processor }
  \begin{table}[htb]
  \centering
  \caption{Area/Delay/Power/Energy results for the processor $P$ built in a $45nm$ process}
  \label{table:atom-power-delay}
  \renewcommand\arraystretch{1.2}
  \setlength{\tabcolsep}{1ex}
  \begin{tabular}{c|c|c|c|c|c}
    \hline
    & Area & Freq & Delay & Power & Energy \\
    & (mm$^2$) & (MHz) & (ms) & (mW) & ($\mu$J) \\
    \hline
    \hline
    A5/1 & 25 & 1600 & 0.12$\mu$s & 2500 & 298.44\,nJ  \\
    \hline
    AES & 25 & 1600 & 0.036 & 2500 & 89.362 \\
    \hline
    FFT & 25 & 1600 & 0.022 & 2500 & 55.64  \\
    \hline
    LPK & 25 & 1600 & 7.90 & 2500 & 19740  \\
    \hline
    RBT & 25 & 1600 & 0.36 & 2500 & 891.89 \\
    \hline
  \end{tabular}
  \end{table}
}


\frame[containsverbatim]{\frametitle{Observations}

\begin{itemize}
\item In terms of energy, the 180nm C-to-RTL circuits are in the
same range as the 45nm processor numbers: for A5, the C-to-RTL circuit
is better by 10X, but for RBT, it is worse by 2X.
\item For a more fair comparison, we should compare 45nm circuits
with the 45nm processor.
\begin{itemize}
\item Since we did not have access to a 45nm library, we scaled the
numbers for the 180nm circuits.
\item Area scaled by 1/16.
\item Delay scaled by 1/4.
\item Power dissipation scaled by 1/2 (we assume that in the 45nm 
circuit, half the power dissipation is due to leakage), so that
energy consumption per completed task will scale by 1/8.
\end{itemize}
\item We compared the scaled numbers with the processor. 
\end{itemize}
}

\frame[containsverbatim]{\frametitle{Comparison of projected 45nm C-to-RTL circuits with 45nm processor}
\begin{table}[htb]
  \centering
  \caption{Area/Delay/Power/Energy RATIOS (processor values relative to the scaled C-to-RTL circuit values)}
  \label{table:Ratios}
  \renewcommand\arraystretch{1.2}
  \setlength{\tabcolsep}{1ex}
  \begin{tabular}{c|c|c|c|c|c}
    \hline
    & Area & Freq & Delay & Power & Energy \\
    \hline
    \hline
    A5/1 & 275.8 & 5.6 & 1.7 & 68.5 & 116.6  \\
    \hline
    AES & 61.5 & 5.6 & 0.34 & 14.8 & 4.9 \\
    \hline
    FFT & 78.4 & 9.6 & 0.57 & 52.6 & 30.3  \\
    \hline
    LPK & 14.8 & 9.6 & 0.84 & 18.3 & 15.3  \\
    \hline
    RBT & 22.2 & 9.6 & 0.15 & 32.7 & 4.7 \\
    \hline
  \end{tabular}
\end{table}
}


\frame[containsverbatim]{\frametitle{Analysis}

\begin{itemize}
\item The C-to-RTL circuits are between 4.7X and 116X more efficient than the processor.
\item The processor delays are lower than those of the C-to-RTL circuit in most
cases (except for A5) but the difference is less than one order of magnitude in
all cases.
\begin{itemize}
\item pipelining of operators and retiming of the C-to-RTL circuits should reduce
this gap.
\end{itemize}
\item To summarize, the C-to-RTL circuits have much better energy efficiency than 
the processor, but have marginally lower performance than the processor.
\end{itemize}

}

\frame[containsverbatim]{\frametitle{Conclusion}
\begin{itemize}
\item We have described a C-to-RTL flow which can be applied to complex
C programs across a wide range of applications.
\item The flow is provably correct by construction (the generated RTL 
is equivalent to the CDFG corresponding to the program from which it
was created).
\item Even in its current state, the flow produces circuits which are
considerably more energy efficient than a state-of-the-art processor.
\item Correctness by construction, energy efficiency and fast turnaround
time imply that this C-to-RTL flow can be an effective alternative to 
the use of embedded processors in implementing complex algorithms in
systems-on-chip.
\item If you wish to try out the flow, please contact madhav@ee.iitb.ac.in
or sameerds@it.iitb.ac.in.
\end{itemize}
}


\frame[containsverbatim]{\frametitle{Thank you!}
}

\end{document}
