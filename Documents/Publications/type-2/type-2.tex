\documentclass[12pt,a4paper]{article}
\usepackage{graphicx}
\usepackage{times}
\usepackage{amsfonts}
\usepackage[margin=0.75in]{geometry}

\title{On the use of petri nets for modelling high-level programs}

\author{Sameer D. Sahasrabuddhe \\ Kavi Arya \\ Madhav P. Desai}

\date{}

\newtheorem{definition}{Definition}

\newtheorem{srule}{Rule}

\begin{document}

\maketitle

\begin{abstract}

  Petri nets provide a powerful mechanism for the modelling of
  concurrent systems. However, because of their generality, it is
  often difficult to prove properties about petri net models unless
  some restrictions are placed on their construction. We consider the
  use of petri nets to model algorithms described in high level
  programming languages. In particular, we introduce a class of live
  and safe petri nets (which we term as {\em Type-2} petri nets) for
  this purpose. This class is defined by a set of refinement rules
  which ensure liveness and safety of the resulting petri net.
  
  Further, we show that Type-2 petri nets support an efficient method
  for concurrency analysis that uses a labelling scheme which is
  analogous to a symbolic simulation of the petri net. Thus, it is
  easy to determine the sets of events (or transitions) which satisfy
  ordering relationships (two transitions are said to be ordered if it
  is impossible for both of them to fire simultaneously). This is
  useful in identifying possibilities for conflict free resource
  sharing when implementing the programs being modelled.

  Despite the restrictions imposed by the production rules, the Type-2
  class of petri nets is powerful enough to represent programs written
  in traditional imperative languages such as C/C++ and also in
  synchronous languages such as Esterel. It has been used as a
  transition step in translating imperative programs to hardware
  descriptions\cite{ahir-thesis}. The resulting high-level synthesis
  flow uses the labelling scheme to reuse hardware resources in the
  absence of timing information.

\end{abstract}

\section{Introduction}
\label{sec:introduction}

Petri nets provide a powerful formalism that has been used to model
and analyse the behaviour of a large class of systems. We consider the
use of petri nets to model the behaviour of algorithms described in
high-level programming languages in order to derive efficient
implementations. We use the petri net to capture the control flow in
the program which is coupled with a data-path (a set of computational
resources) in order to implement the intended behaviour. The petri net
model can be analysed to unveil properties that are important when
deriving an implementation, such as safety and boundedness. In
addition, the analysis can also provide information for efficient
utilisation of resources in the data-path.

However, the complexity of analysing a petri net in general quickly
increases with its size. We adopt the strategy of reducing this
complexity by restricting the structure of the petri net itself. We
propose a class of petri nets (that we call ``Type-2'') that is live
and safe, and can be easily analysed to determine concurrency
relationships between different computations in the program.

The structure of the Type-2 petri net is well suited for representing
the control flow observed in high-level programs --- in particular,
the interaction between concurrency and choice (branches). Threads
that are started within a branch must synchronise before exiting the
branch, and conversely, the targets of a branch encountered in a
thread must be restricted to that thread. The structure of a Type-2
petri net is restricted to guarantee this priority. The same
restrictions enable an efficient labelling scheme that can be used to
analyse the concurrency of operations in the program. Note that the
behaviour allowed by a Type-2 petri net is a generalised version of a
series-parallel net\cite{something}, as can be seen in
Section~\cite{sec:somewhere}.

\section{Definitions}
\label{sec:defintions}

A petri net is a directed bipartite graph with two kinds of nodes ---
\textit{places} and \textit{transitions}. A \textit{marking} is a
function that assigns a non-negative number to each place, indicating
the number of tokens in that place.

\begin{eqnarray*}
  N & = & (P, T, E, M)\\
  \textrm{where } P & = & \textrm{finite set of places}\\
  T & = & \textrm{finite set of transitions}\\
  E & \subseteq & (P \times T) \cup (T \times P)\\
  M & : & P \rightarrow \mathbb{N}
\end{eqnarray*}

The above definition allows at most one edge from a place (transition)
to a transition (place). A \textit{simple place (transition)} is a
place (transition) with one incoming edge and one outgoing place. A
\textit{branch (fork)} is a place (transition) with one incoming edge
and multiple outgoing edges, while a \textit{merge (join)} is a place
(transition) with multiple incoming edges and one outgoing edge. 

A firing sequence is a sequence of transitions $\sigma = [t_{i_0},
t_{i_1}, \ldots]$ in $T$ that may fire in the petri net $N$ starting
at some marking $M$. $L(N)$ is the set of firing sequences possible in
the petri net $N$ starting from the initial marking $M_0$.

\section{Type-1 petri nets}
\label{sec:type-1}

We first define the class of Type-1 petri nets, in terms of a set of
refinement rules (which we call \emph{standard refinements} combined
with a pre-defined basic Type-1 petri net. The class of Type-2 petri
net is a subclass that is defined by an additional refinement rule
introduced in Section~\ref{sec:type-2}.

\begin{definition}
  A \emph{simple cycle} is the petri net $N = (P, T, E, M_0)$ where $P
  = \{p\}$, $T =\{t\}$, $E = \{(p,t), (t,p)\}$ and the
  initial marking $M_0(p) = 1$.
\end{definition}

Clearly, a simple cycle is live and safe. It is defined as the
simplest Type-1 petri net --- any petri net obtained by a sequence of
standard refinements starting with the simple cycle is also a Type-1
petri net. Each standard refinement replaces a simple place or
transition in the original petri net with a subgraph called a
\emph{standard region}. There are two kinds of standard regions ---
\emph{p-type} and \emph{t-type} --- with the following common
properties:

  \begin{enumerate}
  \item Every standard region has a unique \emph{entry} and a unique
    \emph{exit} of the same kind (place or transition) such that:
    \begin{enumerate}
    \item All elements in the region are reachable from the entry.
    \item The exit is reachable from all elements in the region.
    \end{enumerate}
  \item No element in a standard region has an edge incident outside
    the region, with the following exceptions:
    \begin{enumerate}
    \item The entry has exactly one incoming edge incident outside.
    \item The exit has exactly one outgoing edge incident outside.
    \end{enumerate}
  \item Places occuring in a standard region are \textbf{not
      marked} in the initial marking.
\end{enumerate}

\begin{definition} The two kinds of standard regions are as follows:
  \begin{enumerate}
  \item A \emph{p-type standard region} is a standard region where
    every transition is simple, while the entry and exit are both
    places.
  \item A \emph{t-type standard region} is an \emph{acyclic} standard
    region where every place is simple, while the entry and the exit
    are both transitions.
  \end{enumerate}
\end{definition}

A p-type standard region has a structure that allows branching and
loops but not concurrency, while a t-type standard region allows
concurrency but not branching or loops.

\begin{definition}
  A Type-1 petri net is a petri net generated by the following
  production rules:

\setcounter{srule}{-1}
\begin{srule}
  The simple cycle is a Type-1 petri net.
\end{srule}

\begin{srule}
  The petri net obtained from a Type-1 petri net by refining an
  unmarked simple place as a p-type region is also a Type-1 petri net.
\end{srule}

\begin{srule}
  The petri net obtained from a Type-1 petri net by refining a simple
  transition as a t-type region is also a Type-1 petri net.
\end{srule}

\end{definition}

These production rules imply that the place occuring in the original
simple cycle is the only marked place in a Type-1 petri net.

\section{Liveness and Safety in a Type-1 petri net}
\label{sec:live-safe}

We prove that a Type-1 petri net is live and safe by induction over
the structure of the petri net. In particular, we show that the
application of a standard refinement to a Type-1 petri net preserves
liveness and safety in the resulting petri net.

\subsection{p-type standard region}
\label{sec:refinement-place}

Consider a simple place $p \in P$ in a live and safe petri net $N =
(P,T,E,M)$. Let $u,v$ be the predecessor and successor transitions of
$p$ in $N$, i.e, $u,v \in T$ and $(u,p), (p,v)\in E$. Safety implies
that when $u$ fires, it may not fire again until $v$ has fired.
Liveness implies that $u$ can eventually fire again after $v$ has
fired (besides the first firing reachable from the initial marking
$M_0$).

Let $N'$ be the petri net obtained by refining place $p$ as a p-type
region $R$. Then $N'$ is safe if every place in $R$ has at most one
token for any firing sequence starting from $u$. Also, $N'$ is live if
a token that enters region $R$ via transition $u$ can eventually exit
via transition $v$.

\paragraph{Proof:} When $u$ fires, a token is placed in the entry of
the region $R$. Since there are no forks or joins in the region, the
number of tokens within the region is exactly one for any sequence of
firings within the region. Also, no token exits the region until $v$
has fired, and hence $u$ cannot fire again before $v$ does. Hence the
petri net $N'$ is safe. The single token circulates within the region
and can eventually reach the exit, since the exit is reachable from
every element in a standard region. Thus, transition $v$ can
eventually fire after $u$ has fired and hence the petri net $N'$ is
live.

\subsection{t-type standard region}
\label{sec:refinement-transition}

Consider a simple transition $t \in T$ in a live and safe petri net $N
= (P,T,E,M)$. Let $u,v$ be the predecessor and successor places of $t$
in $N$, i.e, $u,v \in P$ and $(u,p), (p,v)\in E$. Safety implies that
when $t$ fires, it will not be enabled again (no token will be placed
$u$) until the token in $v$ has been removed. Liveness implies that
$t$ can eventually be enabled again after the token has been removed
from $v$ (besides the first time when $t$ is enabled in a marking
reachable from $M_0$).

Let $N'$ be the petri net obtained by refining a transition $t$ as a
t-type region $R$. Then $N'$ is safe if every place in $R$ has at most
one token for any firing sequence starting from the entry. Also, $N'$
is live if a token that enters region $R$ via the entry transition
eventually exit via transition the exit transition.

\paragraph{Proof:} The entry of the t-type region fires when a token
is placed in place $u$. Since there are no branches in a t-type
region, at least one token flows along every path starting from the
entry. Thus every transition in the t-type region will fire at least
once, including the exit and hence the petri net $N'$ is live. Since
no new token will be placed in $u$ until a token is removed from $v$,
at most one token (in fact, exactly one token) travels along every
path in the region $R$ for any firing sequence starting with the entry
transition. Hence the petri net $N'$ is safe.

\section{Type-2 petri nets}
\label{sec:type-2}

The Type-1 petri net provides the basic framework for expressing
concurrency and choice encountered in a high-level program. In
addition to these features, a special pattern is required to
represent the effect of branching on values in the program. This
pattern is the \emph{parallel-merge} region that we describe below.

\begin{definition}
  A \emph{Type-2 petri net} is a Type-1 petri net where some of the
  merge places are refined as parallel-merge regions.
\end{definition}

\subsection{Parallel-merge region}
\label{sec:parallel-merge}

When a branch occurs in a program, a variable may be assigned
different values along each path in the branch. The actual value of
the variable at the exit of the branch is decided by the path through
the branch taken at run-time. This is represented by a multiplexer in
the data-path with as many inputs as there are paths merging at a
given merge. Thus, every merge in the control-path may be associated
with one or more such multiplexers to be triggered in parallel. We
represent the control-logic for these multiplexers as a petri net
subgraph called the \emph{parallel-merge region} defined as follows.

\begin{definition}
  Parallel-merge region:
  \begin{enumerate}
  \item A {\em merge fragment} is an acyclic subgraph made of a single
    place with two input edges and one output edge, with a simple
    transition connected to each edge.
  \item An {\em $m$-way merge tree} is a connected acyclic subgraph
    made of $m-1$ merge fragments. The output edge of every merge
    fragment except one is connected to the input edge of some other
    merge fragment through a simple place. The subgraph has $m$ input
    edges and a single output edge. A $2$-way merge tree is a merge
    fragment.
  \item A {\em $k \times m$ parallel-merge region} is a connected
    acyclic subgraph made of the following components:
    \begin{enumerate}
    \item $k$ $m$-way merge trees
    \item $m$ forks with fan-out $k$, such that each output edge of a
      fork is connected to an input edge of a distinct merge tree
      through a simple place
    \item a single join with fan-in $k$, such that each input edge is
      connected to the single output edge of a distinct merge tree
      through a simple place
    \end{enumerate}
  \end{enumerate}
\end{definition}

\subsection{Refining a merge as a parallel-merge region}
\label{sec:refine-merge}

\end{document}

