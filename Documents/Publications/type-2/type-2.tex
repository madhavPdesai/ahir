\documentclass[12pt,a4paper]{article}
\usepackage{graphicx}
\usepackage{times}
\usepackage[margin=0.75in]{geometry}

\title{A class of untimed live and safe petri-nets that supports
  scalable static analysis}

\author{Sameer D. Sahasrabuddhe \\ Kavi Arya \\ Madhav P. Desai}

\date{}

\begin{document}

\maketitle

\begin{abstract}
  \large

  Petri-nets are a compact way of representing concurrent sequences of
  events in a system, where events are represented as transitions
  connected to places that encode the state of the system. We present
  a class of untimed live and safe petri-nets called ``Type-2'' that
  allows a scalable static analysis, without severely affecting the
  expressive power of the petri-net.
  
  The simplest Type-2 petri-net is a cycle consisting of a single
  marked place and a single transition. Larger petri-nets are produced
  by progressive application of production rules that replace an
  unmarked place or a transition with a specific petri-net subgraph
  with no marked places. The initial marking of a Type-2 petri-net is
  thus restricted to the single marked place in the initial cycle. The
  production rules also ensure that every Type-2 petri-net is live and
  safe.

  The restriction on the structure of the Type-2 petri-net
  allows the definition of a simple labelling scheme that effectively
  simulates an execution of the petri-net. The labels of any two
  transitions can be compared in linear time to statically determine
  their relation in the execution of the petri-net. For example, the
  labels can be used determine whether two transitions can fire
  concurrently, thus identifying concurrent events in the system. It
  is also possible to generate an exhaustive listing of sets of
  transitions such that no two transitions in each set can fire
  simultaneously.

  Despite the restrictions imposed by the production rules, the Type-2
  class of petri-nets is powerful enough to represent synchronous
  systems. It has been used as a transition step in translating
  imperative programs to hardware descriptions\cite{ahir-thesis}. The
  resulting high-level synthesis flow uses the labelling scheme to
  optimise the generated hardware by identifying opportunities for
  sharing hardware resources in the absence of timing information.

\end{abstract}

\end{document}
