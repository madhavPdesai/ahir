\documentclass[12pt]{article}
\usepackage{geometry}
\geometry{verbose,a4paper,tmargin=1in,bmargin=1in,lmargin=1in,rmargin=1in}
\usepackage[pdftex]{graphicx}
%\usepackage{fancyvrb}
\usepackage{epstopdf}
%\usepackage{subfigure}
%\usepackage{alltt}
%\usepackage{amssymb}

%\newbox\subfigbox
%\makeatletter
%\newenvironment{subfloat}
%{\def\caption##1{\gdef\subcapsave{\relax##1}}%
%\let\subcapsave\@empty
%\setbox\subfigbox\hbox
%\bgroup}
%{\egroup
%\subfigure[\subcapsave]{\box\subfigbox}}
%\makeatother

\title{Compiler flow for Program to Hardware using AHIR}
\author{Sameer D. Sahasrabuddhe}

\begin{document}

\maketitle

\section{Overview}

The AHIR compiler flow involves a series of passes that result in the
generation of an AHIR specification from a program initially written
in C. The flow is divided into three main stages:

\begin{enumerate}
\item software compiler (LLVM)
\item AHIR generator
\item AHIR linker
\end{enumerate}

The implementation uses a number of freely available C++ software
libraries:

\begin{itemize}
\item LLVM Compiler Framework
\item Boost C++ libraries
\begin{itemize}
\item {\tt program\_options}
\item {\tt string\_algo}
\item {\tt lexical\_cast}
\end{itemize}
\item {\tt libxml++}, the C++ wrapper for the {\tt libxml} XML parser
\end{itemize}

\section{Calling conventions for C functions in AHIR}
\label{sect:calling-conventions}

AHIR, by definition, is independent of the input programming language. 
A specification in AHIR describes a system only in terms of modules
each made of a control and a data path, interacting through a link
layer. In addition, the modules interact with each other through a
separate link layer for control transfer, and pre-assigned memory
locations for data transfer.

A C program is a call-graph, where the vertices are functions and edges
represent function calls. Functions pass data through a function call
in the form of arguments and return values. When translating such a
call-graph to an AHIR specification, each function is represented by a
module, with its own control and data path. The function calls are
translated to control and data interaction between these modules. For
this inter-module communication, we need a mechanism that provides two
services:

\begin{itemize}
\item passing arguments and return values through postboxes
\item implementing a call by passing symbols through the inter-module
      link layer
\end{itemize}

\subsection{A call-site in LLVM}
\label{sect:call-site}

In LLVM, a function call is represented by a {\tt call} instruction,
that has the following properties:

\begin{itemize}
\item a pointer to the called function, or the {\it callee}
\item a list of actual arguments, if present, and their types
\item a return type if present, along with a register in
      the calling function or {\it caller}, to which the return value
      is assigned after the callee returns
\end{itemize}

The instance of a {\tt call} instruction in the body of a function is
called a {\it call-site}. When a call is executed the following
sequence of events happen, where $f$ is the caller and $g$ is the
callee:

\begin{enumerate}
\item prepare actual arguments
\item enter $g$
\item read arguments
\item execute body of $g$
\item prepare return value
\item exit $g$
\item return to the call instruction in $f$
\item read the return value
\end{enumerate}

The actual mechanism through which arguments and return values are
transferred between the caller and the callee, is called the {\it
calling convention} of the target platform. Typical examples are:

\begin{itemize}
\item register based, where the caller stores the arguments in a set
      of registers known to the callee, and expects the return value
      in some particular register.
\item stack based, where the caller pushes arguments onto the stack,
      and expects the return value on the stack after the callee
      returns.
\end{itemize}

There are a number of finer details in these conventions - the order
and locations where arguments occur on stack, for example. Or whether
the callee makes any guarantees about the contents of registers after
it returns.

\subsection{The postbox model}

We specify a similar convention meant for translating C call-graphs to
AHIR modules, that uses {\it postboxes}, or memory locations assigned
to each function, that are valid for the entire duration of execution.
Each function is characterised by a set of input locations, output
locations and internal static allocations:

\begin{table}[htb]
\small
\begin{eqnarray*}
S_F & : & (I_F, O_F, A_F) \mbox{ ... space for function F}\\
I_F & : & \mbox{input locations or arguments}\\
O_F & : & \mbox{output locations or return values (at most one in C)}\\
A_F & : & \mbox{static allocations}
\end{eqnarray*}
\caption{Memory locations assigned to function F}
\label{table:locations-f}
\end{table}

Note that the locations mentioned in Table~\ref{table:locations-f} are
simply addresses that are {\it owned} by a function. They do not
represent the actual data-space addressable by this function. This is
because most of the heavy lifting in C is done by passing pointers as
arguments. Predicting the address space accessed by a function during
the execution of the program requires a detailed memory reference
analysis to track down these argument pointers.

\subsection{A function call in AHIR}
\label{sect:ahir-function-call}

A function call in a C program implies transfer of control and data
across modules in the AHIR specification. To achieve this, a {\tt
call} instruction in the input function is translated into a series of
actions in the corresponding AHIR module. These correspond to the list
of events described in Section~\ref{sect:call-site}, that occur in a
function call. The equivalent AHIR actions are listed below:

\begin{enumerate}
\item a store operation for each actual argument, in the data-path for
      function $f$, using a symbolic address label
\item a transition in the control-path that sends a request symbol
      from alphabet $\Lambda$ to the link layer
\item a load operation for each formal argument in the data-path for
      function $g$
\item a transition in the control-path for function $g$ that waits for
      an input symbol that represents this particular call-site
\item a store operation for the return value in the data-path for
      function $g$
\item a transition that sends an acknowledge symbol for this
      call-site, to the link layer
\item a transition in the control-path for function $f$ that waits for
      an acknowledge symbol from the function $g$
\item a load operation for the return value in the data-path for
      function $f$
\end{enumerate}

\section{Software Compiler}

The LLVM compiler framework is used to provide the first stage of the
flow - a C front-end and an optimising software compiler. LLVM uses
the language front-end from the GCC project, modified to generate
bytecode in the LLVM IR. The GCC front-end is capable of parsing a
number of languages including C and C++. It produces {\tt .bc} files
that contain native LLVM bytecode which can be processed further by
various LLVM utilities.

\begin{figure}[htb]
\centering
\includegraphics[width=4.5in,keepaspectratio]{images/llvm-gcc.eps}
%\caption{LLVM-GCC front-end to translate C to native LLVM bytecode}
%\label{fig:llvm-gcc}
\end{figure}

The LLVM tool ``{\tt opt}'' provides the main optimisation framework. 
{\tt opt} is capable of scheduling various analyses and optmising
passes, based on parameters provided by the user. These passes can
work at the level of function, or the level of a compilation unit,
called ``module'' in LLVM terminology.

\subsection{XML Generator}

An AHIR specification is generated as XML by a pass called {\tt
ModuleGen} implemented in LLVM. This pass is invoked by passing the
``{\tt -module-gen}'' argument to {\tt opt}.

The {\tt ModuleGen} pass lowers a module in the LLVM bytecode to
AHIR-XML in stages. The LLVM bytecode uses an instruction as the
smallest unit of code, grouped into a CFG made of basic blocks, for
each function. {\tt ModuleGen} computes control and data dependences
for these instructions, and constructs a CDFG. This CDFG is translated
to an AHIR specification. The AHIR specification is then printed to a
file in XML format.

\begin{figure}[htb]
\centering
\includegraphics[width=4.5in,keepaspectratio]{images/xml-gen.eps}
%\caption{Generating AHIR-XML from LLVM bytecode}
%\label{fig:xml-gen}
\end{figure}

The {\tt ModuleGen} pass treats each function in the program
separately, generating a control-path, data-path and link layer for
it. The XML consists of a callgraph of functions, and a collection of
control-paths, data-paths and link layers. Each entry in the callgraph
describes the following attributes of a function.

\begin{itemize}
\item number and types of arguments and return value
\item init and fin symbols associated with the function
\item list of callsites that describe other functions called by this
      function
\item labels of associated control path, data path and link layer
\end{itemize}

\subsection{XML Linker}

The specification generated by {\tt ModuleGen} is not complete in
itself. The inter-module link layer (Inter-LL) that translates symbols
in the alphabet $\Omega$ is not present. Also, AHIR requires
predetermined memory locations assigned to each function, to be used
as post-boxes for argument passing, and for static allocations. The
addresses used for these locations do not have actual numeric values
assigned to them when the XML is first generated by {\tt ModuleGen}.

These two details in the XML are filled out by a separate linker, that
parses the generated XML and processes it further. The operations
performed by the linker are described in further detail in
Section~\ref{sect:calling-conventions}. The linked XML specification
is complete in all respects, and represents an AHIR version of the
original C program.

\begin{figure}[htb]
\centering
\includegraphics[width=4.5in,keepaspectratio]{images/ir-link.eps}
%\caption{Linker for functions in AHIR form}
%\label{fig:ir-link}
\end{figure}

The linker produces two files - a linked version of the input AHIR-XML
specification and a map of all the memory locations that have been
assigned addresses while linking. This map is useful for the user,
since it provides information about the location of ``postboxes'' used
by the functions for passing arguments. It also serves as a
description of the address space, that will be useful when designing a
memory subsystem that attempts to optimize parallel accesses to memory
locations.

\section{Two-pass flow: Generate and Link}

The input to the overall flow is a file containing all the C functions
that make up a program. The use of precompiled C libraries may be
enabled using suitable mechanisms in the C front-end. But when
producing the final AHIR specification, the bodies for all the
functions must be available.

\subsection{AHIR-XML Generator}

At the generator stage, each function is processed separately. The
LLVM infrastructure presents the program as an LLVM {\tt module}, that
contains all the functions. The generator iterates over the set of
functions in no particular order, and generates a separate (CP,DP,LN)
triplet from the body of each function.

For each function, the generator produces a preamble in the AHIR-XML
file that lists the following details, to be explained in later
subsections:
\begin{itemize}
\item name of the CP, DP and LN
\item values of the {\tt init} and {\tt fin} symbols
\item static allocations - name, size and address-label
\item formal arguments - name, type and address-label
\item return value - name, type and address-label
\item callsites - called function, req/ack symbol, actual arguments,
      return value
\end{itemize}

A callsite is the occurence of a function call in the body of a
function. The preamble of a function in AHIR-XML provides information
about the callsites occuring in that function.

\subsubsection{Passing data - labels for postboxes}

A number of load-store operations are created at the generator stage,
for passing arguments and return values, as described in
Section~\ref{sect:ahir-function-call}. But the actual addresses for the
corresponding postboxes are not known at this stage. Similarly, the
static allocations within the function bodies do not have valid
addresses as well. The operations involving all these memory locations
use labels instead of numerical addresses.

The address-labels mentioned in these preambles are used in the
various data-paths, as inputs for the load/store operations. With such
labels, each function essentially remains {\it relocatable}. The
functions desribed in such a manner can be used in multiple programs
provided they are eventually assigned valid address spaces.

When a function calls other functions, corresponding callsites are
included in the preamble of the caller. The callsite contains a list
of actual arguments and return value, in the form of address labels,
similar to the formal arguments and return value of a function. These
labels are used within the body of the calling function to generate
corresponding load/store operations. Thus, when a function $f$ calls a
function $g$, store operations are included in the data-path for
function $f$, that refer to the address-labels of the actual arguments
for this callsite. Eventually, the address-labels for the actual
arguments in a callsite and the address-labels for the formal
arguments in the called function must match, so that data is
transferred correctly.

\subsubsection{Passing control}

The control mechanisms of a function call are decoupled in a manner
similar to the data transfer. Each function declares a pair of symbols
that represent the start ({\tt init}) and completion ({\tt fin}) of
execution. Similarly, every callsite declares a pair of symbols for
the request and completion of a function call. The task of mapping a
call request to a function-init and mapping a function-fin to a call
completion symbol is done by the inter-module link layer.

In order to keep function definitions reusable, this assignment of
symbols is also deferred until link time. As a result, the
inter-module link layer that uses the alphabet $\Lambda$ is left
undefined in the AHIR-XML at this stage.

\subsection{AHIR-XML Linker}

The linker is responsible for linking together the functions in the
call-graph of a program. It accepts an AHIR-XML description of
functions, along with a call-graph.

\subsubsection{Addresses for postboxes}

The linker traverses the set of functions in no particular order, and
gathers information about the address-space associated with each
function - formal arguments, return values and static allocations. It
then assigns numerical values to these addresses, keeping in mind the
size of each data-type.

When all the addresses have been assigned values, the linker traverses
the list of callsites in each function. For each callsite, the actual
arguments are assigned the same address as the formal arguments in the
called function. This completes the allocation of memory space to the
address spaces of the functions, which had remained undefined prior to
linking.

\subsubsection{Inter Module Link Layer}

\begin{table}[htb]
\small
\begin{eqnarray*}
inter\;module\;link\;layer & : & \{(function,\{callsite\})\}\\
callsite & : & (req,ack)\\
function & : & (init,fin)
\end{eqnarray*}
\caption{Inter Module Link Layer}
\label{table:inter-ll}
\end{table}

The linker also generates the inter-module link layer, in order to
allow control-transfer across functions. For each function, the linker
generates a list of callsites and the req/ack symbols used by them.
These are placed together to form the specification of the
inter-module link layer. 

As seen from Table~\ref{table:inter-ll}, the inter-module link layer
provides a mapping between pairs of symbols, or {\tt handshakes}. A
function can be called from multiple callsites, but this information
is not available when the function is compiled. Since this information
becomes available only at link time, the task of returning to the
correct caller must be performed by a module mechanism the function. 
Only the inter-module link layer is in a position to ``remember'' the
calling function and return to it when the called function completes
execution. Hence it has to be specified in terms of pairs of symbols,
in order to retain the association within request-acknowledge
handshakes.

\section{Limitations}

The C source code given as input to the frontend must conform to a
number of restrictions. Some of these restrictions are inherent in
AHIR, while others are a result of limitations in the current
implementation.

\subsubsection*{Constraints inherent in AHIR}
\begin{itemize}
\item pointers to functions are not allowed
\item variable number of arguments to functions are not allowed
\item recursion is not allowed; the call-graph must be a DAG
\end{itemize}

\subsubsection*{Limitations of the current implementation}
\begin{itemize}
\item structures are not supported
\item only 32-bit data is supported - smaller widths are treated as
      32-bit, and large widths will give undefined results, possibly a
      compilation error
\item the call-graph is further restricted to be a tree - one function
      can have only one callsite in the entire program
\item the {\tt underscore} ``{\tt \_}'' is not allowed in identifiers;
      it is reserved for generating internal identifiers
\item all the functions must be available in the same module when
      generating the XML - separate compilation for libraries is not
      supported yet
\end{itemize}

\section{SystemC Simulator}

The simulator provides an event-driven model for AHIR, using the
SystemC framework. The simulator is a single SystemC executable that
includes an XML parser. When invoked, the simulator reads the
specification and a memory map in XML format, and creates a SystemC
model of the description.

Typically, a SystemC model consists of a collection of classes
describing various entities. Instances of these classes are created in
the {\tt sc\_main} function, during the elaboration phase. After that,
the simulation is started by the {\tt sc\_start} function.

\begin{verbatim}
int sc_main(int argc, char* argv[])
{
  IRParser parser;
  parser.parse(ir.xml);
  parser.parse(map.xml);

  IrModel *ir = parser.get_ir();

  sc_clock clk();

  connect_signals(ir, clk);

  sc_start(-1) // run indefinitely until some module calls sc_end()
}
\end{verbatim}

The SystemC simulator is program-independent - no SystemC code is
generated that is specific to the current C program. Instead, the
simulator reads an AHIR-XML specification from a file and directly
creates instance of predefined components. It then proceeds to
simulate the components created, producing a trace of events during
execution. The simulator expects two files at invocation:

\begin{itemize}
\item $\langle$program.xml$\rangle$ - an XML file containing the AHIR
      specification
\item $\langle$map.xml$\rangle$ - an XML file describing contents at
      various memory locations
\end{itemize}

The memory map has to be created by hand, based on the list of
locations generated by the linker. It contains test data that is used
as input during simulation, and final data that is to be compared with
the contents of the simulated memory after execution.

%\section{VHDL Generation}



\end{document}

