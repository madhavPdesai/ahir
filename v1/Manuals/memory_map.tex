\documentclass[12pt]{article}
\usepackage{geometry}
\geometry{verbose,a4paper,tmargin=1in,bmargin=1in,lmargin=1in,rmargin=1in}

\usepackage[pdftex]{graphicx}
\usepackage{subfigure}

\usepackage{alltt}
%\usepackage{amssymb}

\title{The Memory Map in XML Format}
\author{Sameer D. Sahasrabuddhe}
\date{November 2006}

\newcommand{\irgen}{{\tt irgen}}
\newcommand{\irlink}{{\tt irlink}}
\newcommand{\irsim}{{\tt irsim}}
\newcommand{\irsyn}{{\tt irsyn}}

\begin{document}

  \maketitle
  
  \section{Introduction}

  The AHIR tools use an XML file to describe the contents of memory as
  seen by the generated AHIR specification. A template map is first
  created by \irlink{} using the locations assigned to various functions
  during the linking process. The map provides a descriptive label for
  each location, along with the type and numerical address. The user
  is expected to fill-in relevant values, using the label.

  \begin{alltt}
  <scalar addr="8" descr="start_formal_0" type="int"></scalar>\end{alltt}

  This instance of the tag ``{\tt <scalar>}'' declares a scalar
  variable described as the first formal argument of the function {\tt
  start}, of type {\tt int}. The address assigned to this variable is
  {\tt 8}. The user may initialise this variable by inserting the
  initial value (say, 10) as contents of the tag:

  \begin{alltt}
  <scalar addr="8" descr="start_formal_0" type="int">10</scalar>\end{alltt}
  

  \noindent{}{\bf NOTE:} The XML file for the memory map that
  \irlink{} generates, is given a name derived from the name of the
  module. For a module {\tt foo}, for example, the memory map is
  called {\tt foo\_map.xml}. This file is created only when \irgen{}
  runs for the first time. If a file of this name already exists,
  \irgen{} will instead creae a file with the name {\tt
  foo\_new\_map.xml}. This file is over-written on subsequent runs,
  with no warnings.

  \section{Structure of the Memory Map}

  The memory map is an XML document, which must start with the tag
  ``{\tt <map>}'', that has two compulsary attributes - ``{\tt id}''
  and ``{\tt size}''. The memory map is divided into two regions - the
  initial values and the final values present in memory.

  \begin{alltt}
  <map id="func" size="42">
  <init>
  <!-- Initial contents of memory -->
  
  </init>

  <!-- first free address: 42
   addresses from this value onwards can be used for external data. -->

  <fin>
  <!-- Final contents of memory -->
  </fin>
  </map>\end{alltt}

  \subsection{Populating the memory map}
  
  The memory map has to be populated only when simulating the
  generated AHIR description of a complete system. For example,
  \irsim{} uses the values to create an {\tt std::map<unsigned,
  bit\_vector>} look-up table which is used to model the external
  memory. In \irsyn{}, the generated test-bench contains an array of
  values, that are written to the system memory through suitable
  access ports before the test bench begins execution.

  In case of synthesisable VHDL code, the test bench is entirely
  ingored, so that the values serve no purpose either. Only the size
  of the memory map is important in order to generate a suitable
  memory component.

  \subsubsection*{The {\tt size} attribute}
  
  The initial memory map is generated by \irlink{}, and contains all
  the locations that are declared in the input C program. These
  include locations for arguments and return value of each function,
  as well as statically declared variables within functions. These
  locations are assigned addresses linearly, starting from {\tt 0}.

  The generated XML file also contains a comment that mentions the
  first free location in the map. The user is free to use addresses
  from there onwards for additional data. When such additional data is
  introduced, the {\tt size} attribute of the map must be set to a
  number that accomodates all the locations declared. This attribute
  is used as a hint by further tools, when creating a memory component
  in the system.

  \subsubsection*{The {\tt init} and {\tt fin} sections}

  The {\tt init} and {\tt fin} sections of the memory map serve two
  purposes:

  \begin{enumerate}
    \item as data entry points for a user during simulation
    \item as information for driver modules, that need to know the
      location of postboxes when sending data to the system during
      actual runs
  \end{enumerate}

\end{document}