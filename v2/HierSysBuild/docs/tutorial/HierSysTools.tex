\documentclass{beamer}
\usepackage{graphics}
\title{Hierarchical System Builder Toolset}
\author{Madhav P. Desai\\ Department of Electrical Engg.\\ IIT Bombay, Mumbai India}
\begin{document}
\maketitle



\frame[containsverbatim]{\frametitle{Overview}

\begin{itemize}
\item The hierarchical system builder toolset is intended for 
assembly of a system in terms of sub-systems.
\item The interfaces of the system and sub-systems are either
{\em pipes} or {\em signals}.   Pipes can be blocking or non-blocking.
\item It is possible to embed register transfer level fragments
in the system description in order to model glue logic.
\end{itemize}
}

\frame[containsverbatim]{\frametitle{Hierachical system description format}

A system description takes the following form:

\begin{verbatim}
$system system_name $library system_library
   $in 
       ... input pipes and signals ...
   $out 
       ... output pipes and signals ...
{
   ... (optional) internals of the system ....
}
\end{verbatim}

The description is kept in a file with extension ``.hsys''.   For example
\begin{verbatim}
system_name.hsys
\end{verbatim}

}

\frame[containsverbatim]{\frametitle{A simple system}

\begin{verbatim}
// comments start with //
$system sysmem $library sysmem_library
   $in 
       // input pipe bus_request, width 32, depth 2
       $pipe bus_request 32 $depth 2
   $out 
       // output pipe bus_response, width 32, depth 2
       $pipe bus_response 32 $depth 2
{
   // empty..
}
\end{verbatim}
This describes a leaf level system which could
have been implemented by translation from Aa to vhdl
by the AHIR V2 toolset.  It's VHDL description  
is supposed to be compiled into the library sysmem\_library.

}

\frame[containsverbatim]{\frametitle{Hierarchy}

A 4-stage shift register is constructed out of two simpler
2-stage shift registers.
\begin{verbatim}
$system shift_reg_4 $library shiftreg_library
   $in 
       $pipe in_data 32 $depth 2
   $out 
       $pipe out_data 32 $depth 2
{
   // internal pipe description..
   $pipe mid_data 32 depth 2
   $instance sr_left shiftreg_library :  shift_reg_2
                in_data => in_data
                out_data => mid_data
   $instance sr_right shiftreg_library :  shift_reg_2
                in_data => mid_data
                out_data => out_data
}
\end{verbatim}
}

\frame[containsverbatim]{\frametitle{Hierarchy: continued}
shift\_register\_2 may itself be build out of 1-stage
shift registers.
\begin{verbatim}
$system shift_reg_2 $library shiftreg_library
   $in 
       $pipe in_data 32 $depth 2
   $out 
       $pipe out_data 32 $depth 2
{
   // internal pipe description..
   $pipe mid_data 32 depth 2
   $instance sr_left shiftreg_library :  shift_reg_1
                in_data => in_data
                out_data => mid_data
   $instance sr_right shiftreg_library :  shift_reg_1
                in_data => mid_data
                out_data => out_data
}
\end{verbatim}
}


\frame[containsverbatim]{\frametitle{Hierarchy: reached the leaf}
\begin{verbatim}
$pipe in_data out_data: $uint<32> $depth 2

$module [shift_reg_1_daemon] $in () $out () $is
{
  $branchblock[loop] {
    $merge $entry loopback $endmerge
    X := in_data
    out_data := X

    $place [loopback]
  }
}
\end{verbatim}

}

\frame[containsverbatim]{\frametitle{Hierarchy: build organization and methodology. See examples/shift\_reg\_4.}
Organize the system into directories which reflect the 
hierarchy.
\begin{verbatim}
Top directory 
    shift_reg_4/

Contains
        shift_reg_2/

which contains
           shift_reg_1/
\end{verbatim}
}


\frame[containsverbatim]{\frametitle{Hierarchy: build organization and methodology continued}
At the leaf directory shift\_reg\_1, we may have an {\bf Aa} project which we
used to generate VHDL  and an aa2c model etc.  At non-leaf directories, we describe
the hierarchy at that level in an hsys file.
\begin{verbatim}
shift_reg_4/
    contains hsys file
    shift_reg_2/
      contains hsys file
      shift_reg_1/
         contains Aa project  
\end{verbatim}
}

\frame[containsverbatim]{\frametitle{Hierarchy: build scripts and make-files at each level}

\begin{verbatim}

shift_reg_4/
    build.sh script generates aa2c of entire hierarchy
    and VHDL at this level.
    shift_reg_2/
       Makefile generates VHDL model at this level
      shift_reg_1/
          Makefile generates aa2c and VHDL of leaf
          shift_reg_1

\end{verbatim}
}

\frame[containsverbatim]{\frametitle{Testbench: See examples/shift\_reg\_4/tb}

\begin{itemize}
\item See tb/tb.c, and compile.sh
\item Use testbench tb\_aa2c to validate the aa2c model.
\item Use testbench tb\_ghdl to validate the VHDL model.
\end{itemize}

}

\frame[containsverbatim]{\frametitle{Glue logic insertion using RTL}

\begin{itemize}
\item In addition to structural descriptions, we can use the
hsys file to describe glue logic using an RTL description.
\item For example, suppose we wish to process the result coming out
of the left shift\_register\_2 by complementing it.
\item We will insert a state machine between the left and
right instances of shift\_register\_2 in the top level
shift\_register\_4.
\item Note: this mechanism cannot be verified by C simulations,
but only by VHDL simulations.
\end{itemize}

}

\frame[containsverbatim]{\frametitle{RTL State Machine}
\begin{verbatim}
$thread complementerFsm
   $in $pipe input_data: $unsigned<32>
   $out $pipe ouput_data: $unsigned<32>
   $variable  data_reg: $unsigned<32>
   $constant z1  : $unsigned<1> := ($unsigned<1>) 0
   $constant o1  : $unsigned<1> := ($unsigned<1>) 1
$default
   //... continued... 
\end{verbatim}
}
\frame[containsverbatim]{\frametitle{RTL State Machine: continued}
\begin{verbatim}
   $now input_data$req  := o1 // default value of pipe read req
   $now output_data$req := z1 // default valu of pipe write req
   <st_empty> {
          $if input_data$ack { // input pipe has data
              data_reg := (~ input_data) // register!
              $goto st_full
          }  
   }
   <st_full> {
          $now output_data$req := o1 // request write to pipe
          $now output_data := data_reg // data to pipe
          $if output_data$ack { // pipe accepts..
                $goto st_empty
          }
          $else { $goto st_full }
   }
   $now $tick
\end{verbatim}
}

\frame[containsverbatim]{\frametitle{RTL State Machine: instantiating}
\begin{verbatim}
$system shift_reg_4 $library shiftreg_library
   $in 
       $pipe in_data 32 $depth 2
   $out 
       $pipe out_data 32 $depth 2
{
   $pipe mid_data_left mid_data_right 32 depth 2
   $instance sr_left shiftreg_library :  shift_reg_2
                in_data => in_data
                out_data => mid_data_left
   $instance sr_right shiftreg_library :  shift_reg_2
                in_data => mid_data_right
                out_data => out_data

   // thread description of complementFsm not shown...

   $string complement_inst: complementFsm 
      input_data => mid_data_left output_data => mid_data_right
}
\end{verbatim}

}

\frame[containsverbatim]{\frametitle{Using RTL State Machines: example shift\_reg\_4\_rtl}
\begin{itemize}
\item Essentially the same as shift\_reg\_4 example, but without aa2c simulation setup.
\item Examine the build and make files to see the differences.
\item Note: only VHDL sim verificationis supported for now.
\end{itemize}
}

\frame[containsverbatim]{\frametitle{Clocks and resets}
\begin{itemize}
\item By default, each block in the system operates on 
clock {\em clk} and reset {\em reset}.  The reset is
active high and synchronized to {\em clk}.
\item It is possible to define alternative clock and
reset signals used in a system.
\item It is possible to map one of these alternative clocks
and resets to the default clock/reset on a subsystem instance.
\end{itemize}
}

\frame[containsverbatim]{\frametitle{Declaring clock-like and reset-like signals}
\begin{verbatim}
$system Top $library TopLib
   $in
     $pipe A
     $signal $clk EXTCLK 1 $depth 1
     $signal $reset EXTRESET 1 $depth 1
   $out
     $pipe B
{
}
\end{verbatim}
Here, we are stating that the input signal EXTCLK is clock-like
and the input signal RESET is reset-like.
}

\frame[containsverbatim]{\frametitle{Using clock-like and reset-like signals}
\begin{verbatim}
// ... as above ... 
{
 $pipe TMP $clk => EXTCLK $reset =>  EXTRESET
 $instance i0  S1:Stage1
         ..
         $clk => EXTCLK
         $reset => EXTRESET

 $instance i1  S2:Stage2
         ...
         $clk => EXTCLK
         $reset => EXTRESET

\end{verbatim}
}

\frame[containsverbatim]{\frametitle{Using clock-like and reset-like signals}
\begin{itemize}
\item We have declared EXTCLK as clock-like and EXTRESET as reset-like.
\item Pipe TMP and instance i0 will operate with EXTCLK connected
to their default clock input and EXTRESET connected to their default
reset input.
\item We have generated a local clock EXTCLK\_REPEATED using an
RTL machine.
\item Instance i0 will operate with EXTCLK\_REPEATED connected
to its default clock input and EXTRESET connected to its default
reset input.
\item Only a clock-like signal can be mapped to the default clock.
\item Only a reset-like signal can be mapped to the default reset.
\end{itemize}
}

\frame[containsverbatim]{\frametitle{Example: using clock-like and reset-like signals}
See the example clocks\_and\_resets.

}

\frame[containsverbatim]{\frametitle{Summary}

\begin{itemize}
\item Convenient way to assemble large systems which use pipes and signals
to communicate.
\item Build script: buildHierarchicalModel.py can generate aa2c and vhdl models.
\item Lots of error checking: dangling connections, width mismatch, direction mismatch.
\item Can embed glue RTL in description files 
\begin{itemize}
\item If you use embedded RTL, then only vhdl sims are possible for verification.
\end{itemize}
\item Tried and proven tools in AJIT, NAVIC project!
\end{itemize}

}
\end{document}
