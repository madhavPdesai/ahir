
\documentclass{article}

\title{vC Language Reference Manual}
\author{Madhav Desai \\ Department of Electrical Engineering \\ Indian Institute of Technology \\
	Mumbai 400076 India}

\newcommand{\Aa}{{\bf Aa}~}
\newcommand{\vC}{{\bf vC}~}

\begin{document}
\maketitle

\section{Introduction}

{\bf vC} is a language used for the description
of a virtual circuit which is decomposed into
control, data and storage aspects.

A \vC system consists of modules and memory spaces.
Each module may itself contain memory spaces.  Further,
each module has a {\em control-path} and a {\em data-path}.
The control-path is a Petri-net constructed using
a set of rules (as described in \cite{ref:SameerPhD}).
The data-path is essentially a directed graph of {\em operator } nodes
communicating with each other and the outside world using {\em wires}.

A memory-space stores data and provides access to this
data through load/store operators.
The \vC system also  contains {\em pipes} which
are first-in-first out queues. These pipes can
be accessed from a data-path using input/output
operators.

In the rest of this document, we outline the
structure and the syntax of the \vC language,
and also describe the semantics of a \vC
program, especially the execution model and
the behaviour of the equivalent system.


\section{Program structure}


A program (equivalently, a system) in \vC consists of a sequence of
declarations and module definitions.  
\begin{verbatim}
program := ( module-definition | 
              memory-space-definition | 
                pipe-declaration | 
                  attribute-specification )*
\end{verbatim}

A module in \vC is the basic unit of compilation, and
has the following structure
\begin{verbatim}
$module [module-name] 
{
   input-arguments (can be empty)
   output-arguments (can be empty)
   memory-space-definitions (optional)
   pipe-declarations (optional)
   control-path (required)
   data-path (required)
   control-data-links (optional)
   attribute-specifications (optional)
}
\end{verbatim}

Thus, a module has a name, has input and output arguments,
pipes, memory spaces, a control-path and a data path.
Pipes declared in the module are visible only in the
module body.   

\section{Variables} \label{sec:Objects}

Variables in  a \vC system can belong to the following classes:
\begin{itemize}
\item {\bf Argument} variable declared at the module-level.
\item {\bf Wire} variables declared in the data-path.
\item {\bf Pipe} variables declared at the program level or at the module level.
\end{itemize}

Each variable has a type.  In \vC a fairly rich array of
types is supported, as described in Section \ref{sec:Types}.

{\bf Wire} variables are declared in the data-path, and
serve as connectors between operators.  A wire can have
at most one operator node which drives it, but can
be read by any number of operators.

{\bf Pipe} variables declared at the program scope
are globally visible first-in-first-out (FIFO) buffers that can 
be accessed from any module in the program.  A pipe
is a FIFO buffer with a specified number of
slots.  A write to a pipe succeeds
only if there is an empty slot in the FIFO (a successful
write uses up an empty slots.  A read
from the pipe succeeds only if there is at
least one occupied slot in the FIFO (a successful read
creates an empty slot in the FIFO). 


\section{The control-path} \label{sec:Controlpath}
 

The control-path is a Petri-net which consists of
places and transitions.  In \vC this Petri-net
is described hierarchically in terms of 
control-path elements. 
\begin{itemize}
\item A single transition is a control-path element. It
fires when all of its predecessors have (released) a token.
After it fires, a token is released to each of
its successors.
\begin{verbatim}
$T [fire]
\end{verbatim}
This defines a transition with name $fire$. 
\item A single place is a control-path element.  When
any of its predecessors releases a token, the place
takes it and forwards it to one of its successors.
\begin{verbatim}
$P [loopback]
\end{verbatim}
\item A {\em series region} is a sequence of control
path elements.  The predecessor of each element is
the previous element and the successor is the next
element in the sequence.  
The predecessor of
the first element is an implicit entry transition
of the series region, and the successor of the last
element is an implicit exit transition of
the series region.
\begin{verbatim}
;;[chain] {
       $T [req]
       $T [ack]
}
\end{verbatim}
\item A {\em parallel region} is a sequence of control
path elements.  The implicit entry transition
is the predecessor of all the elements, while the
implicit exit transition is the successor of
all the elements.
\begin{verbatim}
||[pardo] {
       ;;[job_1] { ... }
       ;;[job_2] { ... }
}
\end{verbatim}
\item A {\em branch region} is a sequence of control
path elements with additional statements which describe
the connections between this sequence (branches, merges).
\begin{verbatim}
<>[branch_example] {
       $P [Enter]
       Enter <-| ($entry)
       Enter |-> (job_1)
       ;;[job_1] { ... }
       ;;[job_2] { ... }
       $P [Chk]
       $P [End]
       Chk <-| (job_1)
       Chk |-> (job_1 job_2)
       End <-| (job_2)
       End |-> ($exit)
}
\end{verbatim}
In this example, the keywords \$entry and \$exit correspond
to implicit entry and exit transitions associated with the branch block.
The 
\begin{verbatim}
<-|
\end{verbatim}
 specifies that a token leaving the right will enter
the left (place) and 
\begin{verbatim}
|->
\end{verbatim}
signifies that a token entering the left
(place) will enter one of the elements in the list on the right.
\item A {\em fork region} is a sequence of control path
elements with additional statements which describe 
forking and joining.
\begin{verbatim}
::[fork_example] {
       $entry &-> (job_1 job_2)
       ;;[job_1] { ... }
       ;;[job_2] { ... }
       ;;[job_3] { ... }
       $T [f12]
       f12 <-& (job_1 job_2)
       f12 &-> (job_3)
       $exit <-& (job_3)
}
\end{verbatim}

In this example, the keywords \$entry and \$exit correspond
to implicit transitions associated with the branch block.
The
\begin{verbatim}
 <-&
\end{verbatim}
 specifies that after a token leaves from every
element in the right list, a token will enter
the left (transition) and the
\begin{verbatim}
&->
\end{verbatim}
signifies that a token leaving the left
(transition) will be replicated so that a
token enters each of the elements in the list on the right.
\item A pipelined-fork block defines a fork block which can support
several active instantiations at the same time.
\begin{verbatim}
:|: [<name>] {
   same-structure-as-fork
   but-can-have-marked-joins
}
\end{verbatim}
This uses a special join called a marked-join
\begin{verbatim}
jT o<- (s1 1 s2 1)
(s1 can be $entry)
\end{verbatim}
Which says that jT is a join of s1 and s2, with a pre-condition
that s1 is assumed to have fired and s2 is assumed to have fired
before time began.  As a result it is possible for jT to fire
before s1 and s2 fire in normal operation. 
\end{itemize}

The control-path itself is a series region. It syntax is
illustrated below:
\begin{verbatim}
$CP {
   ... sequence of control-path elements --
}
\end{verbatim}

\subsection{Special constructs in the control path}

Some special constructs are used in the control path.
\begin{itemize}
\item Transition merge:  This is specified as
\begin{verbatim}
$transition_merge [merge_name] (p q r) (s)
\end{verbatim}
Where p, q, r are transition, and s fires whenever any 
of p,q,r fire.
\item  Phi-sequencer:   This is specified
as
\begin{verbatim}
$phisequencer [phiseq_name] :  
    trigger_1
       src_1_sample_req src_1_sample_ack
       src_1_update_req src_1_update_ack
    trigger_2
       src_2_sample_req src_2_sample_ack
       src_2_update_req src_2_update_ack:
    phi_sample_start phi_sample_completed
       phi_update_start phi_update_completed :
    trigger_1_sample_req  trigger_2_sample_req:
       phi_mux_ack
\end{verbatim}
and it describes the control connections between the Phi operator and
its sources.
\item Loop terminator: This is specified as
\begin{verbatim}
$terminate (loop_exit loop_taken loop_body_done) 
              (loop_back loop_terminate)
\end{verbatim}
The loop terminator looks at the result of the do-while condition
check via the input transitions loop\_exit and loop\_taken, 
and uses the loop\_body\_done transition to determine the
number of transitions in flight and hence to trigger the
next do-while iteration (firing of loop\_back) or to
decide that the loop has terminated.
\end{itemize}

\section{Types} \label{sec:Types}

Types in \vC can be either scalar types or composite types.  They
are always attached to objects such as pipes, wires and input/output
arguments (types are not declared separately!).

Scalar types can be one of 
\begin{itemize}
\item Integers:  An integer type
has a width parameter and is specified as 
\begin{verbatim}
$int<width>
\end{verbatim}
The width parameter can be any positive number.
Values corresponding to this type are to be
viewed as a binary sequence of the specified width.
\item Pointers:  A pointer is an unsigned
integer with an unspecified width which
is associated with a memory space.
The width of the pointer is determined by the
implementation at the point of conversion from
\vC to VHDL.  For example:
\begin{verbatim}
$pointer< memory_space_1 > 
\end{verbatim}
\item Floats: A float is parametrized by
two integers, the width of the exponent,
and the width of the mantissa.  
The specification is
\begin{verbatim}
$float<exponent,mantissa>
\end{verbatim} %$
where the exponent and mantissa must be positive
integers.  The float is represented by 
a word with $exponent+mantissa+1$ bits
(with the additional bit needed for the sign).
The standard IEEE 754 float and double
precision representations correspond
to
\begin{verbatim}
$float<8,23> 
\end{verbatim}
and
\begin{verbatim}
$float<11,52> 
\end{verbatim}
respectively.  Currently, these are the
only float types that are supported.
\end{itemize}

Composite types in \vC can be either
array types or record types.
\begin{itemize}
\item 
Array types in \vC have the form
\begin{verbatim}
$array [d1] $of <element-type-spec>
\end{verbatim} %$
The value of $d1$ must
be a positive integers, and element-type-spec
must refer to a type.
For example, 
\begin{verbatim}
$array [10] of $array [10] $of $int<32>
\end{verbatim}
is an array type whose
elements are one dimensional arrays of 32-bit
unsigned integers.
\item
Record types in \vC have the form
\begin{verbatim}
$record <type-1> <type-2> ... <type-n>
\end{verbatim}
An element of such a record type is
an aggregate whose first element is of type
type-1, second element is of type type-2 and so on.
For example:
\begin{verbatim}
$record <$uint<32> > <$array [10] $of $uint<32> > 
\end{verbatim}
\end{itemize}


\section{The Data-path} \label{sec:Datapath}

The Data-path is a directed graph of operator nodes
interconnected with wires.
A wire is declared as
\begin{verbatim}
$W [wname] : <wire-type>
\end{verbatim}
For example:
\begin{verbatim}
$W [reg_out] : $int<33>
\end{verbatim}

One can also have constant wires which are declared
as
\begin{verbatim}
$constant $W [wname] : <wire-type> := <wire-value>
\end{verbatim}
where wire-value is a constant literal.  A constant wire can only have 
scalar type.

Operator nodes can be differentiated based on two criteria.
Based on the number of inputs, they can be classified
as unary, binary, variable argument and mutliple-output
operators.  Based on their protocol of interaction with
the control path, they can be classified as those
with a split-protocol (separate start-complete request/ack
handshakes) or a unitary protocol (single request/ack handshake).


Operator nodes are instantiated by the syntax
\begin{verbatim}
($guard (~)? guard-wire)? <op-code> [operator-name] ( <input-wires> )  ( <output-wires>)
\end{verbatim}
For example, an add operator node instantiation is 
\begin{verbatim}
+ [wadd] (a b)  (c)
\end{verbatim}
The add operation takes two inputs and produces a single output.
Some other variants:
\begin{verbatim}
$guard (p) + [wadd1] (a b)  (c)
\end{verbatim}
signifies that the wadd1 operation is actually executed only if
the wire $p$ is 1.
\begin{verbatim}
$guard (~p) + [wadd2] (a b)  (c)
\end{verbatim}
signifies that the wadd2 operation is actually executed only if
the wire $p$ is 0.


Other binary operators with two inputs and a single
output are (all binary operators have a split protocol): 
\begin{verbatim}
op-code operation             
-       subtract             
*       multiply            
/       divide             
<<      shift-left        
>>      shift-right-logical 
$S>>    shift-right-arithmetic
==      equal              
$S>$S   signed greater-than
$S>=$S  signed greater-than-or-equal
$S<$S   signed less-than
$S<=$S  signed less-than-or-equal
>       unsigned greater-than
>=      unsigned greater-than-or-equal
<       unsigned less-than
<=      unsigned less-than-or-equal
!=      not-equal
|       logical-bitwise-or
&       logical-bitwise-and
^       logical-bitwise-xor
~|      logical-bitwise-nor
~&      logical-bitwise-nand
~^      logical-bitwise-xnor
&&      concatenate
\end{verbatim}

Operators with a single input and a single
output are (all have a split-protocol):
\begin{verbatim}
~       logical-not
:=      assign-operation
$S:=$U  signed-to-unsigned-conversion
$U:=$S  unsigned-to-signed-conversion
$S:=$S  signed-to-signed-conversion
$S:=$F  signed-to-floating-point-conversion
$F:=$S  floating-point-to-signed-conversion
$U:=$F  unsigned-to-floating-point-conversion
$F:=$U  floating-point-to-unsigned-conversion
$F:=$F  float-to-float-conversion
\end{verbatim}

There are some special operators which
need some explanation.
\begin{itemize}
\item The {\em bit-select} operator has two inputs and
one output.  The syntax is
\begin{verbatim}
[] [selbit] (a i) (b)
\end{verbatim}
Wire b will get the value of the $i^{th}$ most significant bit
of a.  This operator has a split protocol. 
\item  The {\em PHI} operator has a variable number of inputs
and one output.  The syntax is
\begin{verbatim}
$phi [oname] (u1 u2 u3 .. un) (y)
\end{verbatim}
This operator has a split protocol, but there are as many
sample-req/sample-ack pairs as there are inputs.  There is a
single update-req/update-ack pair.  The output y
is assigned to an input for which a request has been made.
\item The {\em slice} operator has three inputs, the first
of which is a wire, and the next two must be constants.
It has a single output:
\begin{verbatim}
[:] [cname] (a 5 3) (c)
\end{verbatim}
Then c = a[5 downto 3].
\end{itemize} 

Finally, there are some special operators, all
of which have split protocols.
\begin{itemize}
\item The {\em load} and {\em store} operators are used to access
a memory space.
\begin{verbatim}
$load [fetch1] $from memory_space_1 (addr_wire) (data_wire)
$store [write1] $to memory_space_1 (addr_wire data_wire) ()
\end{verbatim}
\item The {\em input} and {\em output} operators are used
to read from and write to pipes (these have a unitary protocol).
\begin{verbatim}
$ioport $in  [R1] (pipe_to_be_read) (wire_to_be_written_into)
$ioport $out [W1] (wire_to_be_read) (pipe_to_be_written_into)
\end{verbatim}
\item The {\em call} operator is used to send arguments to
another module and wait until results are returned (it has
a split protocol).
\begin{verbatim}
$call [call_id] $module foo (in1 in2 .. inN) (out1 out2 .. outM)
\end{verbatim}
The number of input and output arguments must match those
on module foo.
\end{itemize}

{\bf Note: All operators can be guarded!}

\section{Control-data links} 

The control path and the data path work in
a coordinated fashion.  Transitions in the control
path are associated with request and acknowledge
signals on operators in the data path.  % TODO.

\section{Memory Spaces}

Each memory space provides a certain amount of 
storage with a well defined access width.  % TODO.

\section{Syntax} \label{sec:Syntax}

The syntax for \vC follows the following
principles
\begin{itemize}
\item Most keywords begin with the \$ sign.
\item The region between \{ and \} defines a new scope.
\item Elements are space separated (no semicolons at all).
\end{itemize}

The parser is implemented using 
an LL(k) parser (written as rules to be parsed by antlr2 \cite{ref:antlr2}).
The grammar for the parser is (using the EBNF notation) given in 
the html file {\bf vcParser.html} which is part of this distribution. 
The set of tokens recognized by the lexical analyzer (or lexer).
is available in the html file {\bf vcLexer.html}.

\begin{thebibliography}{99}
\bibitem{ref:SameerPhD}
Sameer D. Sahasrabuddhe,
``A competitive pathway from high-level programs to hardware,''
Ph.D. thesis, IIT Bombay, 2009.

\bibitem{ref:antlr2}
http://www.antlr2.org.
\end{thebibliography}

\end{document}
