\documentclass[12pt]{article}

\title{Some examples of the use of Aa relative to C}
\author{Madhav Desai \\ Department of Electrical Engineering \\ Indian Institute of Technology \\
	Mumbai 400076 India}

\newcommand{\Aa}{{\bf Aa}~}
\newcommand{\vC}{{\bf vC}~}

\begin{document}
\maketitle

We describe some examples of functional descriptions in {\bf C} and their equivalents in \Aa.


\section{Finding the maximum of $32$ numbers}

Consider a system which reads in 32 numbers from an input FIFO and stores them into an array.
Then the system finds the smallest item in the array, as well as the index
of the smallest item.  These two values are written out to an output FIFO.

In C the program could be written as follows.
\begin{verbatim}
void minItem() {
    uint32_t items[32];
    uint32_t I;

    // read loop
    for(I = 0;  I < 32; I++)
       items[I] = read_uint32("input_pipe");

    // min item loop
    uint32_t smallest = 0xffffffff;
    uint32_t smallest_index = 33;
    for(I = 0;  I < 32; I++)
    {
       if(items[I] <= smallest)
       {
           smallest = items[I];
           smallest_index = I;
       }
    }
   
    // write out.
    write_uint32("out_pipe", smallest);
    write_uint32("out_pipe", smallest_index);
}
\end{verbatim}

The almost exact equivalent of this C program in \Aa could
be written as follows.

\begin{verbatim}
$pipe input_pipe output_pipe: $uint<32>
$module [minItem] $in () $out () $is
{
   $storage items: $array[32] $of $uint<32>

   // read loop
   $branchblock[readLoop] {
       $merge $entry loopback
          $phi I := $zero<8> $on $entry nI $on loopback
       $endmerge

       $volatile nI := (I + 1)
       items[I] := input_pipe

       $if (I < 31) $then
           $place[loopback]
       $endif 
   
       // min item loop
       $merge $entry min_item_loopback
          $phi J := $zero<8> $on $entry 
                                    nJ $on min_loopback
          $phi min_val := (~$zero<32>) $on $entry 
                                    n_min_val $on min_loopback
          $phi min_index := ($bitcast ($uint<32>) 33) $on $entry 
                                    n_min_index $on min_loopback
       $endmerge
       $volatile nJ := (J + 1)
       X := items[J]
       $if (X <= min_val) $then
           n_min_val := X
           n_min_index := J
       $endif   

       $if (J < 31) $then
           $place[loopback]
       $endif 
   }  (n_min_val => MINVAL n_min_index => MININDEX)
  
   out_pipe := MINVAL
   out_pipe := MININDEX
}
\end{verbatim}

\section{Using double loops: matrix multiplication}

We read in a square matrix $A$ and calculate the 
matrix $A^2$.  The C program for this could be as follows.
\begin{verbatim}
void matrixSquarer() {
   uint32_t I, J, K;
   uint32_t A[32][32];
   uint32_t ASQ[32][32];

   // read matrix
   for (I = 0; I < 32; I++) {
      for (J = 0; J < 32; J++) {
        A[I][J] = read_uint32("input_pipe");
      }
   }
   // matrix multiply
   for (I = 0; I < 32; I++) {
      for (J = 0; J < 32; J++) {
         uint32_t tmp = 0;
         for (K = 0; J < 32; J++) {
            tmp += A[I][K]*A[K][J];
         }
         ASQ[I][J] = tmp;
      }
   }
   // send out.
   for (I = 0; I < 32; I++) {
      for (J = 0; J < 32; J++) {
        write_uint32("output_pipe", A[I][J]);
      }
   }
}
\end{verbatim}

The equivalent \Aa program could be as follows.
\begin{verbatim}
$module [matrixSquarer] $in () $out () $is {
   $storage A B: $array[32][32] $of $uint<32>

   // read loop
   $branchblock[readLoop] {

       ////////////////////   read loop /////////////////////
       $merge $entry outer_loopback
          $phi I := $zero<8> $on $entry nI $on outer_loopback
       $endmerge

       $merge $entry inner_loopback
          $phi J := $zero<8> $on $entry nJ $on inner_loopback
       $endmerge

       $volatile nI := (I + 1)
       $volatile nJ := (J + 1)
       A[I][J] := input_pipe

       $if (J < 31) $then
           $place[inner_loopback]
       $endif 
       $if (I < 31) $then
           $place [outer_loopback]
       $endif
   
       ////////////////////   matrix-multiply loop /////////////////////
       $merge $entry mm_outer_loopback
          $phi II := $zero<8> $on $entry nII $on mm_outer_loopback
       $endmerge

       $merge $entry mm_inner_loopback
          $phi JJ := $zero<8> $on $entry nJJ $on mm_inner_loopback
       $endmerge

       $volatile nII := (II + 1)
       $volatile nJJ := (JJ + 1)

       $merge $entry mm_dotp_loopback
          $phi K := $zero<8> $on $enry nK $on mm_dotp_loopback
          $phi TMP := $zero<32> $on $entry nTMP $on mm_dotp_loopback
       $endmerge
 
       nTMP := (TMP + (A[II][K] * A[K][JJ])

       $if (K < 31) $then
           $place [mm_dotp_loopback]
       $else
           B[II][JJ] := nTMP
       $endif

       $if (JJ < 31) $then
          $place [mm_inner_loopback]
       $endif 

       $if (II < 31) $then
          $place [mm_outer_loopback]
       $endif_

       ////////////////////   send loop /////////////////////
       $merge $entry w_outer_loopback
          $phi III := $zero<8> $on $entry nIII $on w_outer_loopback
       $endmerge

       $merge $entry w_inner_loopback
          $phi JJJ := $zero<8> $on $entry nJJJ $on w_inner_loopback
       $endmerge

       $volatile nIII := (III + 1)
       $volatile nJJJ := (JJJ + 1)
       output_pipe := B[III][JJJ]

       $if (JJJ < 31) $then
           $place[w_inner_loopback]
       $endif 
       $if (III < 31) $then
           $place [w_outer_loopback]
       $endif
   
   } 
}
\end{verbatim}


\section{Using wide SSA variables to implement a shift regiser}

In this system, we read in 32-bit values from an input pipe and pass them through
an 8 stage shift register.   Outgoing values are written into
an output pipe.  Instead of using storage variables, we use wide
SSA variables to maintain the state of the shift register.

\begin{verbatim}
$pipe input_pipe output_pipe: $uint<32>
$module [shiftReg8] $in () $out () $is
{
   $branchblock[loop] {
        $merge $entry loopback
           $phi SR := $zero<256> $on $entry nSR $on loopback
           $phi V  := $zero<8>   $on $entry nV $on loopback
        $endmerge
        X := input_pipe

        // shift in from the right.
        nSR := ($concat ($slice SR 223 0) X)
        nV  := ($concat ($slice V 7 1) $one<1>)

        $if (V [] 7) $then
            output_pipe := ($slice SR 255 224)
        $endif
   }
}
\end{verbatim}




\end{document}


