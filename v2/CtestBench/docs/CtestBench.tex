\documentclass{article}

\title{On verifying AhirV2 generated VHDL using software testbenches}
\author{Madhav Desai \\ Department of Electrical Engineering \\ Indian Institute of Technology \\
	Mumbai 400076 India}

\newcommand{\Aa}{{\bf Aa}~}
\newcommand{\vC}{{\bf vC}~}

\begin{document}
\maketitle


The AhirV2 tool chain can be used to convert parts of a C program to VHDL
(essentially, some of the functions in a program are mapped to VHDL).
To verify the resulting VHDL, one would like to simulate it in a
VHDL simulator (such as Modelsim from Mentor Graphics).  The most
natural way to do this is to use the original program itself
as a testbench for this purpose.

\begin{itemize}
\item Stubs are created for the set of functions which are mapped to  
VHDL by the AhirV2 flow.
\item The software testbench is compiled and linked with these stubs.
\item Whenever a stub function is called, it tries to connect with
a server created by the VHDL simulation process.
\item The VHDL simulation process listens for calls from the stubs
and exchanges data between the stubs and the actual VHDL being simulated.
\end{itemize}


We assume that you have a {\bf Release} directory of the AhirV2 flow
installed.   The Release/bin subdirectory contains the binaries for
the translation of LLVM-byte-code down to a VHDL description. 
The Release/vhdl directory contains VHDL library elements that 
are instantiated in this VHDL description.  The Release/CtestBench 
directory contains the libraries and headers that you will need.  
Please see Release/CtestBench/examples/helloworld for an 
illustration of the whole flow.
\end{document}
