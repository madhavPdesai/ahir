\documentclass{article}

\title{Aa Language Reference Manual}
\author{Madhav Desai \\ Department of Electrical Engineering \\ Indian Institute of Technology \\
	Mumbai 400076 India}

\newcommand{\Aa}{{\bf Aa}~}

\begin{document}
\maketitle

\section{Introduction}

The {\bf Aa} language is intended to be used to
describe an algorithm so that the control flow
in the description can be modeled by a petri net
of a specific class (the type-2 petri-net as
introduced in \cite{ref:SameerPhD}).  A program in
\Aa is also a description of a system which reacts with 
its environment through input and output ports and 
through message queues.

In the rest of this document, we outline the
structure and the syntax of the \Aa language,
and also describe the semantics of an \Aa
program, especially the execution model and
the behaviour of the equivalent system.


\section{Program structure}


A program in \Aa consists of a sequence of
declarations and module definitions.  
\begin{verbatim}
program := ( module-definition | program-object-declaration )*
\end{verbatim}
Declarations at the program level 
can belong to one of three classes:
\begin{itemize}
\item {\bf Storage} objects declared in the program
are globally visible variables that can be read from
anywhere in the program and can be written from anywhere
in the program.
\item {\bf Pipe} objects declared in the program are
globally visible first-in-first-out queues that can 
be read from anywhere in the program. 
written into by at most one module in the program.
\begin{itemize}
\item A pipe can be written into by at most one module
in the program, and can be read from by at most one module
in the program.
\item A pipe that is only written into in the program, but
is not read from, is assumed to have a destination outside
the program.
\item A pipe that is only read from in a program but not
written into, is assumed to have a source outside the program.
\end{itemize}
\item {\bf Constant} objects are globally visible and have
initial values which can never be altered in the program.
Thus, an assignment to a constant is an error.
\end{itemize} 

A module in \Aa is the basic unit of compilation, and
has the following structure
\begin{verbatim}
module-definition :=  
                 module-specifier module-name  
                 module-input-arguments
                 module-output-arguments
                 module-object-declarations
                 { <module-body> }
\end{verbatim}
Thus, a module has a name, has input and output arguments,
declares objects and consists of a sequence of statements.
Objects (storage/pipe/constant) declared in the module are visible only in the
module body. 

The module body itself is a sequence of statements.
\begin{verbatim}
module-body := (statement)+
\end{verbatim}
The control flow implied is serial in nature.  The
module starts executing when it gets a token.
This token is passed to the first statement in the
list which executes and releases the token when it
finishes execution, and the token  is passed serially down the
list of statements.  When the last statement finishes 
executing, the token passes out of the module.

The region between \{ and \} is termed a {\bf scope}.  Thus, 
the module description defines a scope.  In this case, 
the scope has a label which is the same as the module name.
As we shall see later, the statements in the statement sequence
may in turn define scopes, which may be labeled or anonymous.
Each scope defines a name-space.  The visibility
rules and access mechanisms between scopes will be described
in Section \ref{sec:Scopes}

The control flow in an \Aa module can be visualized
in terms of the flow of tokens.
A module is invoked when it receives a token, and 
this token flows down the sequence of statements.
Whenever a statement receives a token, it executes,
and when it has finished executing, the token is
released to the next statement. 

\section{Statements}
 
Statements in the module-body can be simple statements
or block statements.

\subsection{Simple Statements}

Simple statements can be one of the following:
\begin{itemize}
\item An {\bf assignment} statement is of the form
\begin{verbatim}
target-ref := expression
\end{verbatim}
where {\em target-ref} either specifies a declared object
or an undeclared name.  If it is an undeclared name, then
it is an implicit variable that is attached to the
assignment statement.  No other statement can write to this
variable, but any statement can read from this variable
subject to scoping rules.  Thus, every implicit variable
is such that exactly one statement defines its value.
When control flow reaches an assignment statement, it
evaluates the expression, updates the value of the target
and passes the control flow forward (we will see what this
means in the sequel).
Some examples of simple statements are 
\begin{verbatim}
a  :=  ($not b)
b  :=   (c + (d + e))
\end{verbatim}
etc.  
More details of the syntax are provided in 
Section \ref{sec:Syntax}.
\item A {\bf call} statement is of the form
\begin{verbatim}
call-spec module-name input-argument-values output-argument-targets
\end{verbatim}
The call statement thus specifies a module to which control flow
is to be passed.  The call statement may itself define new implicit
variables through its specified targets (and thus is the only
statement that can modify these newly defined implicit variables).
When control flow reaches the call statement, it forwards the
token to the called module, and finishes when the token exits
the called module.  Recursive calls and cyclical dependencies between
modules through call statements are not permitted in the \Aa language.
Here is an example of a call statement
\begin{verbatim}
// pass control to a module named foo
// foo has two inputs and three outputs
$call foo (p,q)  (r,s,t)
\end{verbatim}
More details of the syntax are provided in 
Section \ref{sec:Syntax}.
\item The {\bf null} statement does nothing, and just passes the
token onwards.  This is how it looks like
\begin{verbatim}
$null
\end{verbatim}
\end{itemize}

\subsection{Block Statements}

Block statements can be used to describe complex
control flows, and consist of a sequence of statements.
Block statements are of
the following types:
\begin{itemize}
\item {\bf Series} block statements are of the form
\begin{verbatim}
series-block-statement :=
             series-block-specifier block-name
             { 
                declarations
                sequence-of-statements
             }
\end{verbatim}
The behaviour is similar to the module, in that the control
token flows serially down the sequence of statements.
Here is an example:
\begin{verbatim}
$seriesblock [s1] 
{
   $declare $storage b $uint<32>
   b := a
   b := ( $mux (a > c)  (b+c) (b-c))
   q := ( b * 2)
}
\end{verbatim}
In this example, the variable $b$ was declared as
a storage variable in scope s1, and hence was legally
able to be the target of multiple assignment statements.
\item {\bf Parallel} block statements are of the form
\begin{verbatim}
parallel-block-statement :=
             parallel-block-specifier block-name
             { 
                declarations
                sequence-of-statements
             }
\end{verbatim}
When a token enters the parallel block statement, it
is replicated into as many tokens as there are statements in
the sequence, and all statements are started in parallel.
When all statements have finished and released their token,
the parallel block statement
ends and a single token exits the parallel block statement (this
is essentially a fork followed by a join).
Here is an example
\begin{verbatim}
$parallelblock [p1] 
{
   b := a + c
   d := a - c
}
\end{verbatim}
The two statements will start in parallel, and the block 
will finish when both have finished.  The
order in which the two statements are executed
is {\bf not} specified.
\item {\bf Fork} block statements are generalizations of the
parallel block region in the sense that they
allow the programmer to express complex fork-join interactions.
They have the following structure:
\begin{verbatim}
fork-block-statement :=
             fork-block-specifier block-name
             {
                declarations
                sequence-of-statements
             }
\end{verbatim}
The control flow is similar to the parallel block region,
in the sense that all statements in the sequence are started
in parallel, and the fork block statement ends when all statements
in the sequence have ended.  The difference is that
fork blocks can have an additional statement which
allows the programmer to provide additional synchronization
points.  These statements are termed {\bf join} statements
and have the form
\begin{verbatim}
join-statement :=
         join-specifier list-of-labels 
          fork-specifier  list-of-statements
\end{verbatim}
The meaning of this statement is that it waits until all statements
in the list-of-labels have finished, then starts the list-of-statements
(in parallel) and finishes.  This essentially defines a node in a directed graph,
with the arcs corresponding to statements in the statement-lists.  
The join indicates the arcs (or statements) 
from whom tokens have to arrive.  When all tokens have arrived
on the incoming arcs, tokens are sent along the outgoing arcs specified
in the list of statements following the optional fork specifier.  There is an implicit fork
at the entry point to the fork block statement and an implicit join at the
end of the fork block statement.  It is thus easy to describe
an arbitrary directed graph with arbitrary forks and joins.  The only
restriction is that this directed graph must be acyclic!
Here is an example:
\begin{verbatim}
$forkblock [f1] {
  a := (b+c)
  $seriesblock [s1] { ... }
  $seriesblock [s2] { ... }
  $seriesblock [s3] { ... }
  $join s1 s2 $and $fork 
      $seriesblock [s4] { ... }
  $join s3 s4
}
\end{verbatim}
In this example, the first assignment statement as well as s1, s2, s3 are 
started in parallel. When both s1 and s2 have finished, s4 is started and
when s3, s4 have joined and when the first assignment statement has
finished, the forkblock f1 finishes execution.
\item {\bf Branch} block statements are constructs which allow the programmer
to describe arbitrary sequential branching behaviour.  They 
describe a control flow in which a single token 
is active within the block.  The movement of this token is
controlled by special flow control statements.
They are structured 
as follows
\begin{verbatim}
branch-block-statement :=
             branch-block-specifier block-name
             { sequence-of-statements}
\end{verbatim}
The sequence of statements appearing in a branch block
consist of
\begin{itemize}
\item Simple statements or Block statements.
\item Switch statements: A switch statement has
the form
\begin{verbatim}
switch-statement :=
         switch-spec switch-expression
           ( expr-value  statement-to-be-executed )*
           ( default statement-to-be-executed )?
            
\end{verbatim}
The switch-expression is checked, and depending on its
value, one of the alternative statements is selected.
Thus, the incoming token to the switch statement is 
passed to one of the alternatives.  For example,
\begin{verbatim}
$switch a $when 0 $then a1 := (b + c)
          $when 1 $then a2 := (b - c)
          $default $null
$endswitch
\end{verbatim}
The control token is routed to the appropriate
choice and if the token is not routed out 
of the block (by the {\bf takeroute} statement 
described below), then the token passes to the
statement following the switch. 
\item If statements: An if statement has the form
\begin{verbatim}
if-statement :=
       if-spec test-expression 
           true-statement
       (else
           false-statement)?
       
\end{verbatim}
For example, 
\begin{verbatim}
$if (a != 0) $then
   q := (r + s)
   t := 0
$else
   qdash :  (r - s)
$endif
\end{verbatim}
\item Routing statement: The {\bf takeroute} statement
defines a labeled path along which it sends the
incoming control token.  For example,
\begin{verbatim}
$takeroute [fastpath]
\end{verbatim}
means that the incoming token is not sent out to the
next statement in the sequence, but is sent out along
a path labeled ``fastpath''.  The token will be captured
either by a {\bf merge} statement, or will exit the
containing branch block.
\item Merge statements: A merge statement is
specified as 
\begin{verbatim}
merge-statement :=
          merge-spec label-list merge-assignments
\end{verbatim}
which is to be interpreted as follows.  The labels in the label list
refer to paths defined by ``takeroute'' statements within the branch block.
Whenever a token arrives along any of these paths,
the merge statement starts and executes a series of special assignments
which multiplex values into new variables based on which arc the
token arrived from.  The merge statement then releases its token to
the next statement. 
\end{itemize}

Here is an example of a branch block constructed in this manner
\begin{verbatim}
$branchblock [b1] 
{
  $merge loop $entry // $entry is the path 
                     // into the block

       $phi q := 0 on $entry r on loop 
                    // q is defined by where 
                    // the token came from
  r := q + 1
  $if r < 10 $then 
       $takeroute [loop]
  $endif
}
\end{verbatim}

\end{itemize}

Thus, the program is constructed as a collection of modules,
each of which is a sequence of statements.  The use of series,
parallel, fork and branch block
statements enables the programmer to describe a highly concurrent
structured system with complex branching behaviour.  
The resulting control flow structure is a
type-2 petri-net with provable liveness and safety properties \cite{ref:SameerPhD}.

\section{Scoping Rules}

An \Aa program is made of modules which in turn contain statements
and so on.  The program thus has a hierarchy of scopes with each
scope being contained in another (except for the program itself,
which is not contained in any scope).

The rules for scoping are as follows:
\begin{enumerate}
\item A variable (declared or implicit) defined in a scope is visible
in all descendant scopes.
\item A reference to a variable $b$ in a scope X is resolved by checking
whether the variable is defined in that scope, and if not found there,
by checking in the scope that contains the scope X, and so on.
\item A scope can read from variables that are defined in descendant
scopes.  
\item A scope can read from variables that are defined in ancestor scopes.
\end{enumerate}

A variable which is to be read in a statement may be specified 
as follows
\begin{verbatim}
a            look for variable a in current scope; 
             if not found look in the parent.

:a           same as the previous case

^:a          look for variable a defined in the parent of
             the current scope.

%p:a         look for variable a defined in the child scope 
	     with label p (child scope of the current scope).

^^:a         look for variable a defined in the parent of
             the parent of the current scope.

%p%q:a       look for variable a defined in the child scope 
             with label q of the child scope with label p of 
             the current scope

a[10]        look for a storage variable a defined in the 
             current module.  If not found, look for it in
             the program scope.  If found in either of the places,
             access the corresponding element of the array.
\end{verbatim}
Thus, a generic variable reference has the form 
\begin{verbatim} 
    scope-id : variable-reference 
\end{verbatim} 
and the scoping rules forbid a scope from accessing variables
which are not defined in either an ancestor or a descendant
of the scope.

\section{Syntax}

The syntax for \Aa is described by the following lexical analysis and
parsing rules (written as rules to be parsed by antlr2 \cite{ref:antlr2},
a LL(k) parser).

This is the set of tokens recognized by the lexical analyzer (or lexer).
\begin{verbatim}


Definition of lexer AaLexer, which is a subclass of CharScanner.

/** Lexer nextToken rule:
 *  The lexer nextToken rule is synthesized from all of the user-defined
 *  lexer rules.  It logically consists of one big alternative block with
 *  each user-defined rule being an alternative.
 */
mMODULE 
|	mDECLARE 
|	mDEFAULT 
|	mSTORAGE 
|	mPIPE 
|	mCONSTANT 
|	mSERIESBLOCK 
|	mPARALLELBLOCK 
|	mFORKBLOCK 
|	mBRANCHBLOCK 
|	mPLACE 
|	mSWITCH 
|	mENDSWITCH 
|	mIF 
|	mENDIF 
|	mOF 
|	mON 
|	mTHEN 
|	mELSE 
|	mFORK 
|	mJOIN 
|	mMERGE 
|	mENDMERGE 
|	mENDJOIN 
|	mWHEN 
|	mENTRY 
|	mEXIT 
|	mFIN 
|	mIN 
|	mOUT 
|	mIS 
|	mASSIGN 
|	mCALL 
|	mPHI 
|	mCOLON 
|	mSEMICOLON 
|	mCOMMA 
|	mQUOTE 
|	mHASH 
|	mASSIGNEQUAL 
|	mEQUAL 
|	mNOTEQUAL 
|	mLESS 
|	mLESSEQUAL 
|	mQUESTION 
|	mGREATER 
|	mGREATEREQUAL 
|	mIMPLIES 
|	mSHL 
|	mSHR 
|	mLBRACE 
|	mRBRACE 
|	mLBRACKET 
|	mRBRACKET 
|	mLPAREN 
|	mRPAREN 
|	mPERCENT 
|	mCARET 
|	mPLUS 
|	mMINUS 
|	mMUL 
|	mDIV 
|	mNOT 
|	mOR 
|	mAND 
|	mXOR 
|	mNOR 
|	mNAND 
|	mXNOR 
|	mMUX 
|	mUINT 
|	mINT 
|	mFLOAT 
|	mPOINTER 
|	mNuLL 
|	mARRAY 
|	mCAST 
|	mUINTEGER 
|	mUFLOAT 
|	mWHITESPACE 
|	mSINGLELINECOMMENT 
|	mSCOPE_IDENTIFIER 
|	mSIMPLE_IDENTIFIER 
mMODULE
	:	"$module" 
	;

mDECLARE
	:	"$declare" 
	;

mDEFAULT
	:	"$default" 
	;

mSTORAGE
	:	"$storage" 
	;

mPIPE
	:	"$pipe" 
	;

mCONSTANT
	:	"$constant" 
	;

mSERIESBLOCK
	:	"$seriesblock" 
	;

mPARALLELBLOCK
	:	"$parallelblock" 
	;

mFORKBLOCK
	:	"$forkblock" 
	;

mBRANCHBLOCK
	:	"$branchblock" 
	;

mPLACE
	:	"$place" 
	;

mSWITCH
	:	"$switch" 
	;

mENDSWITCH
	:	"$endswitch" 
	;

mIF
	:	"$if" 
	;

mENDIF
	:	"$endif" 
	;

mOF
	:	"$of" 
	;

mON
	:	"$on" 
	;

mTHEN
	:	"$then" 
	;

mELSE
	:	"$else" 
	;

mFORK
	:	"$fork" 
	;

mJOIN
	:	"$join" 
	;

mMERGE
	:	"$merge" 
	;

mENDMERGE
	:	"$endmerge" 
	;

mENDJOIN
	:	"$endjoin" 
	;

mWHEN
	:	"$when" 
	;

mENTRY
	:	"$entry" 
	;

mEXIT
	:	"$exit" 
	;

mFIN
	:	"$fin" 
	;

mIN
	:	"$in" 
	;

mOUT
	:	"$out" 
	;

mIS
	:	"$is" 
	;

mASSIGN
	:	"$assign" 
	;

mCALL
	:	"$call" 
	;

mPHI
	:	"$phi" 
	;

mCOLON
	:	':' 
	;

mSEMICOLON
	:	';' 
	;

mCOMMA
	:	',' 
	;

mQUOTE
	:	'"' 
	;

mHASH
	:	'#' 
	;

mASSIGNEQUAL
	:	":=" 
	;

mEQUAL
	:	"==" 
	;

mNOTEQUAL
	:	"!=" 
	;

mLESS
	:	'<' 
	;

mLESSEQUAL
	:	"<=" 
	;

mQUESTION
	:	'?' 
	;

mGREATER
	:	">" 
	;

mGREATEREQUAL
	:	">=" 
	;

mIMPLIES
	:	"=>" 
	;

mSHL
	:	"<<" 
	;

mSHR
	:	">>" 
	;

mLBRACE
	:	'{' 
	;

mRBRACE
	:	'}' 
	;

mLBRACKET
	:	'[' 
	;

mRBRACKET
	:	']' 
	;

mLPAREN
	:	'(' 
	;

mRPAREN
	:	')' 
	;

mPERCENT
	:	'%' 
	;

mCARET
	:	'^' 
	;

mPLUS
	:	'+' 
	;

mMINUS
	:	'-' 
	;

mMUL
	:	'*' 
	;

mDIV
	:	'/' 
	;

mNOT
	:	"$not" 
	;

mOR
	:	"$or" 
	;

mAND
	:	"$and" 
	;

mXOR
	:	"$xor" 
	;

mNOR
	:	"$nor" 
	;

mNAND
	:	"$nand" 
	;

mXNOR
	:	"$xnor" 
	;

mMUX
	:	"$mux" 
	;

mUINT
	:	"$uint" 
	;

mINT
	:	"$int" 
	;

mFLOAT
	:	"$float" 
	;

mPOINTER
	:	"$pointer" 
	;

mNuLL
	:	"$null" 
	;

mARRAY
	:	"$array" 
	;

mCAST
	:	"$cast" 
	;

mUINTEGER
	:	mDIGIT ( mDIGIT )* 
	;

protected mDIGIT
	:			'0'..'9' 
	;

mUFLOAT
	:	'.' mUINTEGER 'E' 
		(	'+' 
		|	'-' 
		) 
		mUINTEGER 
	;

mWHITESPACE
	:	(	' ' 
		|	'\t' 
		|	'\f' 
		|	'\r' 
		|	'\n' 
		) 
		
	;

mSINGLELINECOMMENT
	:	( "//" ( ~'\n' )* '\n' ) 
	;

mSCOPE_IDENTIFIER
	:	'.' ( mSIMPLE_IDENTIFIER '.' )* 
	;

mSIMPLE_IDENTIFIER
	:	mALPHA 
		(	mALPHA 
		|	mDIGIT 
		|	'_' 
		)* 
	;

protected mALPHA
	:			'a'..'z' 
	|			'A'..'Z' 
	;

\end{verbatim}


As one can see, all keywords in the \Aa language begin with \$.

The grammar for the parser is (using the EBNF notation) given below. 

\begin{verbatim}


Definition of parser AaParser, which is a subclass of LLkParser.


aA_Program
	:	(	( aA_Module ) 
		|	( DECLARE ( aA_Object_Declaration )+ ) 
		)* 
	;


aA_Module
	:	MODULE aA_Label aA_In_Args aA_Out_Args IS LBRACE 
		(	DECLARE ( aA_Object_Declaration )+ 
		|	
		) 
		aA_Atomic_Statement_Sequence RBRACE 
	;


aA_Object_Declaration
	:	( aA_Storage_Object_Declaration ) 
	|	( aA_Constant_Object_Declaration ) 
	|	( aA_Pipe_Object_Declaration ) 
	;


aA_Label
	:	( LBRACKET ) ( SIMPLE_IDENTIFIER ) ( RBRACKET ) 
	;


aA_In_Args
	:	IN LPAREN ( aA_Interface_Object_Declaration )* RPAREN 
	;


aA_Out_Args
	:	OUT LPAREN ( aA_Interface_Object_Declaration )* RPAREN 
	;


aA_Atomic_Statement_Sequence
	:	( aA_Atomic_Statement )+ 
	;


aA_Interface_Object_Declaration
	:	( aA_Object_Declaration_Base ) 
	;


aA_Atomic_Statement
	:	aA_Assignment_Statement 
	|	aA_Call_Statement 
	|	aA_Null_Statement 
	|	aA_Block_Statement 
	;


aA_Assignment_Statement
	:	( aA_Object_Reference ) ASSIGNEQUAL ( aA_Expression ) 
	;


aA_Call_Statement
	:	CALL SIMPLE_IDENTIFIER aA_Argv_In aA_Argv_Out 
	;


aA_Null_Statement
	:	NuLL 
	;


aA_Block_Statement
	:	aA_Series_Block_Statement 
	|	aA_Parallel_Block_Statement 
	|	aA_Fork_Block_Statement 
	|	aA_Branch_Block_Statement 
	;


aA_Object_Reference
	:	(	(	( PERCENT SIMPLE_IDENTIFIER )* 
			|	( CARET )+ 
			) 
			COLON 
		|	
		) 
		( SIMPLE_IDENTIFIER ) ( LBRACKET aA_Expression RBRACKET )* 
	;


aA_Expression
	:	(	( aA_Constant_Literal_Reference ) 
		|	( aA_Object_Reference ) 
		|	( aA_Unary_Expression ) 
		|	( aA_Binary_Expression ) 
		|	( aA_Ternary_Expression ) 
		) 
	;


aA_Argv_In
	:	LPAREN ( aA_Object_Reference )* RPAREN 
	;


aA_Argv_Out
	:	LPAREN ( SIMPLE_IDENTIFIER )* RPAREN 
	;


aA_Series_Block_Statement
	:	SERIESBLOCK 
		(	aA_Label 
		|	
		) 
		LBRACE 
		(	DECLARE ( aA_Object_Declaration )+ 
		|	
		) 
		aA_Atomic_Statement_Sequence RBRACE 
	;


aA_Parallel_Block_Statement
	:	PARALLELBLOCK 
		(	aA_Label 
		|	
		) 
		LBRACE 
		(	DECLARE ( aA_Object_Declaration )+ 
		|	
		) 
		aA_Atomic_Statement_Sequence RBRACE 
	;


aA_Fork_Block_Statement
	:	FORKBLOCK 
		(	aA_Label 
		|	
		) 
		LBRACE 
		(	DECLARE ( aA_Object_Declaration )+ 
		|	
		) 
		aA_Fork_Block_Statement_Sequence RBRACE 
	;


aA_Branch_Block_Statement
	:	BRANCHBLOCK 
		(	aA_Label 
		|	
		) 
		LBRACE 
		(	DECLARE ( aA_Object_Declaration )+ 
		|	
		) 
		aA_Branch_Block_Statement_Sequence RBRACE 
	;


aA_Fork_Block_Statement_Sequence
	:	(	( aA_Join_Fork_Statement ) 
		|	( aA_Atomic_Statement ) 
		)+ 
		
	;


aA_Join_Fork_Statement
	:	JOIN ( SIMPLE_IDENTIFIER )* 
		(	AND FORK aA_Atomic_Statement_Sequence 
		|	
		) 
		ENDJOIN 
	;


aA_Branch_Block_Statement_Sequence
	:	( (	( aA_Merge_Statement ) 
			|	( (	aA_Switch_Statement 
					|	aA_If_Statement 
					) ) 
			|	( aA_Atomic_Statement ) 
			|	( aA_Place_Statement ) 
			) 
			)+ 
	;


aA_Merge_Statement
	:	MERGE 
		(	( SIMPLE_IDENTIFIER ) 
		|	( ENTRY ) 
		)+ 
		(	( aA_Phi_Statement )+ 
		|	
		) 
		ENDMERGE 
	;


aA_Switch_Statement
	:	SWITCH aA_Expression ( WHEN aA_Constant_Literal_Reference THEN aA_Branch_Block_Statement_Sequence )* 
		(	DEFAULT aA_Branch_Block_Statement_Sequence 
		|	
		) 
		ENDSWITCH 
	;


aA_If_Statement
	:	IF ( aA_Expression ) THEN aA_Branch_Block_Statement_Sequence 
		(	ELSE aA_Branch_Block_Statement_Sequence 
		|	
		) 
		ENDIF 
	;


aA_Place_Statement
	:	PLACE aA_Label 
	;


aA_Block_Statement_Sequence
	:	( aA_Block_Statement )+ 
	;


aA_Phi_Statement
	:	PHI SIMPLE_IDENTIFIER ASSIGNEQUAL ( aA_Expression ON 
			(	( SIMPLE_IDENTIFIER ) 
			|	( ENTRY ) 
			) 
			)+ 
	;


aA_Constant_Literal_Reference
	:	(	PLUS 
		|	MINUS 
		|	
		) 
		(	(	( UINTEGER ) 
			|	( UFLOAT ) 
			) 
		|	( LESS 
				(	( UINTEGER )+ 
				|	( UFLOAT )+ 
				) 
				GREATER ) 
		) 
		
	;


aA_Unary_Expression
	:	LPAREN 
		(	( ( NOT ) ( aA_Expression ) ) 
		|	( CAST LPAREN ( aA_Type_Reference ) RPAREN ( aA_Expression ) ) 
		) 
		RPAREN 
	;


aA_Binary_Expression
	:	LPAREN aA_Expression aA_Binary_Op aA_Expression RPAREN 
	;


aA_Ternary_Expression
	:	LPAREN MUX aA_Expression ( aA_Expression ) ( aA_Expression ) RPAREN 
	;


aA_Type_Reference
	:	( aA_Scalar_Type_Reference ) 
	|	( aA_Array_Type_Reference ) 
	;


aA_Binary_Op
	:	( OR ) 
	|	( AND ) 
	|	( NOR ) 
	|	( NAND ) 
	|	( XOR ) 
	|	( XNOR ) 
	|	( SHL ) 
	|	( SHR ) 
	|	( PLUS ) 
	|	( MINUS ) 
	|	( DIV ) 
	|	( MUL ) 
	|	( EQUAL ) 
	|	( NOTEQUAL ) 
	|	( LESS ) 
	|	( LESSEQUAL ) 
	|	( GREATER ) 
	|	( GREATEREQUAL ) 
	;


aA_Storage_Object_Declaration
	:	( STORAGE aA_Object_Declaration_Base ) 
	;


aA_Constant_Object_Declaration
	:	( CONSTANT aA_Object_Declaration_Base ) 
	;


aA_Pipe_Object_Declaration
	:	( PIPE aA_Object_Declaration_Base ) 
	;


aA_Object_Declaration_Base
	:	( SIMPLE_IDENTIFIER ) ( aA_Type_Reference ) 
		(	ASSIGNEQUAL aA_Constant_Literal_Reference 
		|	
		) 
	;


aA_Scalar_Type_Reference
	:	( aA_Uint_Type_Reference ) 
	|	( aA_Int_Type_Reference ) 
	|	( aA_Float_Type_Reference ) 
	|	( aA_Pointer_Type_Reference ) 
	;


aA_Uint_Type_Reference
	:	UINT LESS UINTEGER GREATER 
	;


aA_Int_Type_Reference
	:	INT LESS UINTEGER GREATER 
	;


aA_Float_Type_Reference
	:	FLOAT LESS UINTEGER COMMA UINTEGER GREATER 
	;


aA_Pointer_Type_Reference
	:	POINTER LESS UINTEGER GREATER 
	;


aA_Array_Type_Reference
	:	ARRAY ( LESS UINTEGER GREATER )+ OF ( aA_Scalar_Type_Reference ) 
	;


\end{verbatim}



\section{Examples}

Here is a very simple program
\begin{verbatim}
// an array of 32-bit unsigned integers.
$storage mem $array<1024> of $uint<32> 
$module [sel_mod]
   $in (a $uint<32> b $uint<10>)
   $out (c $uint<32>)
$is
{
   c := (mem[b] + a)
   mem[b] := c
}
\end{verbatim}
This consists of a single module, which accumulates
a value into an array position.


A little bit more complicated:
\begin{verbatim}
// an array of 32-bit unsigned integers.
$storage mem $array<1024> of $uint<32> 

// module returns the sum of mem[I] from
// I=low to I=high
$module [sum_mod]
    $in (low $uint<10> high $uint<10>)
    $out (sum $uint<32>)
$is
    $storage I $uint<10>
    $storage J $uint<10>
    $storage sum1 $uint<32>
    $storage sum2 $uint<32>
{
 
   d := (high-low)  
   mp := ((high-low)/2)

   I := low
   J := (mp + 1)

   \branchblock[trivcheck]
    {
       $merge $entry
       $if (d > 0) $then
           // do two summations in parallel
           // parallel summations
           \parallelblock[parsum]
           {
               \branchblock[sb1]{
                    merge $entry loopback
                       $phi s := 0 on $entry %forward:s on loopback
                    $if (I < mp) $then
                        \seriesblock[loopback]{
                            I := (I+1)
                            s := $mux (I == 0)  0 (s + mem[I])
                         }
               } 
               \branchblock[sb2]{
                    merge $entry loopback
                       $phi s := 0 on $entry %forward:s on loopback
                    $if (J < high) $then
                        \seriesblock[loopback]{
                            J := (J+1)
                            s := $mux (J == (mp+1))  0 (s + mem[J])
                         }
               } 
           }
 
           // combine results from parallel statement above
           snontriv := (%parsum%sb1:s  + %parsum%sb2:s)
           $takeroute [nontrivsum]
      $else

           // summation is trivial 
           striv := mem[low]
           $takeroute [trivsum]
      $merge nontrivsum trivsum
         // which sum do you pick? depends on which path was taken
         $phi sum := snontriv on nontrivsum strivsum on trivsum
      $exit
    }

    // bring sum up to the surface.
    sum := %trivcheck:sum
}
\end{verbatim}

\end{document}
 







