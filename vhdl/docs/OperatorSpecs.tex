\documentclass{beamer}
\usepackage{epsfig}
\usepackage{alltt}
\begin{document}

\frame[containsverbatim]{\frametitle{Datatypes: type boolean}
We will use the  VHDL boolean type to model control
signals.  The following array type will also be
needed
\begin{verbatim}
type BooleanArray is array(integer range <>) of boolean;
\end{verbatim}
}

\frame[containsverbatim]{\frametitle{Datatypes: type natural, integer}
We will use the VHDL {\bf integer} and {\bf natural} types
to model generics.  The following array types will also
be needed
\begin{verbatim}
type IntegerArray is array(integer range <>) of integer;
type NaturalArray is array(integer range <>) of natural;
\end{verbatim}
}

\frame[containsverbatim]{\frametitle{Datatypes:  2D std\_logic array {\bf StdLogicArray2D}}
A two dimensional array of std\_logic, defined as
\begin{verbatim}
type StdLogicArray2D 
  is 
  array(integer range <>, integer range <>) 
  of std_logic;
\end{verbatim}
}

\frame[containsverbatim]{\frametitle{Datatypes: type {\bf ApInt}}
The LLVM arbitrary precision integer. 
The VHDL equivalent is
\begin{verbatim}
type ApInt is array(integer range <>) of std_logic;
\end{verbatim}
Object declarations will fix the width.  For example
\begin{verbatim}
signal a: ApInt(5 downto 0);
\end{verbatim}
}

\frame[containsverbatim]{\frametitle{Datatypes: type {\bf ApIntArray}}
An array of ApInt.  The VHDL equivalent is
\begin{verbatim}
type ApIntArray is 
  array(integer range <>, integer range <>) 
  of std_logic;
  -- 8 ApInts, each of width 32.
signal a: ApIntArray(0 to 7, 31 downto 0);
\end{verbatim}
The range for the second dimension will always be of the type
"high downto low".
}


\frame[containsverbatim]{\frametitle{Functions: extract/insert on {\bf ApIntArray}}

The following extract and insert functions will be used to
access elements of the array.
\begin{verbatim}
function 
  Extract (constant x: ApIntArray; 
           constant idx: integer)
  return ApInt;
procedure
  Insert (x: out ApIntArray; 
          y: in ApInt; 
          idx: in integer);
\end{verbatim}

The parameter idx must be an element of \begin{verbatim} x'range(1). \end{verbatim}

}

\frame[containsverbatim]{\frametitle{Operators: conversion operators on {\bf ApInt}}
\begin{verbatim}
component ApInt_S_1
             port (x: in ApIntArray; 
                   clk,reset: in std_logic; 
                   req: in BooleanArray;
                   ack: out booleanArray;
                   z: out ApIntArray);
end component;
\end{verbatim}
This is the component corresponding
to single input shared operations.
The component can be mapped using the following
configurations:
\begin{verbatim}
ApIntToApIntSigned, ApIntToApIntUnsigned, ApIntNot.
\end{verbatim}
}

\frame[containsverbatim]{\frametitle{Operators: {\bf ApInt} arithmetic/logical operators }
\begin{verbatim}
component ApInt_S_2
	     generic (colouring: NaturalArray);
             port (x,y: in ApIntArray; 
                   clk,reset: in std_logic; 
                   reqR,reqC: in BooleanArray;
                   ackR,ackC: out BooleanArray;
                   z: out ApIntArray);
end component;
\end{verbatim}
This is the component corresponding
to two input shared operations.
The following configurations can be applied to this component:
\begin{verbatim}
ApIntAdd, ApIntSub, ApIntAnd, ApIntOr, ApIntXor, ApIntMul,
ApIntSHL, ApIntLSHR, ApIntASHR (sign bit extension)
ApIntEq, ApIntNe, ApIntUgt, ApIntUge, ApIntUlt,
ApIntUle, ApIntSgt, ApIntSge, ApIntSlt, ApIntSle.
\end{verbatim}
For the shift operators, the argument y is interpreted as
the shift-amount (assumed to be unsigned!).
}

\frame[containsverbatim]{Operators: \frametitle{{\bf ApInt} binary operators with one constant operand }
If the second argument is statically known, then the following component should be
used.
\begin{verbatim}
component ApInt_S_2_C
                generic (y: ApInt; colouring : NaturalArray);
                port ( x: in ApIntArray; 
                   clk,reset: in std_logic; 
                   reqR,reqC: in BooleanArray;
                   ackR,ackC: out BooleanArray;
                   z: out ApIntArray);
end component;
\end{verbatim}
The following configurations are similarly defined for
this component:
\begin{verbatim}
ApIntAddC, ApIntSubC, ApIntAndC, ApIntOrC, ApIntXorC, 
ApIntMulC, ApIntSHLC, ApIntLSHRC, ApIntASHRC ApIntEqC, 
ApIntNeC, ApIntUgtC, ApIntUgeC, ApIntUltC,ApIntUleC, 
ApIntSgtC, ApIntSgeC, ApIntSltC,ApIntSleC.
\end{verbatim}
}


\frame[containsverbatim]{Operators: \frametitle{Operators: {\bf ApInt} multiplexor operators }
\begin{verbatim}
component ApIntPhi 
                   port(x,y: in  ApInt; 
                        reqx,reqy: in boolean;
                        z: out ApInt; 
                        ack: out boolean;
                        clk,reset: in bit);
end component;
component ApIntSelect 
                   port(x,y,sel: in  ApInt; 
                        req: in boolean;
                        z: out ApInt; 
                        ack: out boolean;
                        clk,reset: in bit);
end component;
\end{verbatim}
These operators are special, and are not shared.
}

\frame[containsverbatim]{\frametitle{Operators: {\bf ApInt} branch operator}
\begin{verbatim}
component ApIntBranch 
             port (condition: in ApInt; 
                   clk,reset: in std_logic; 
                   req: in Boolean;
                   ack0: out Boolean;
                   ack1: out Boolean);
end component;
\end{verbatim}
This operator is special and is not shared.
}

\frame[containsverbatim]{\frametitle{Operators: load components {\bf ApLoadReq} }
\begin{verbatim}
component ApLoadReq
          generic(width: NaturalArray; 
                  colouring : NaturalArray;
                  suppress_immediate_ack: BooleanArray);
          port(addr: in StdLogicArray2D;
               req: in BooleanArray;
               ack: out BooleanArray;
               mreq: out std_logic_vector;
               mack: in std_logic_vector;
               maddr: out std_logic_vector;
               mtag: out std_logic_vector;
               clk,reset: in std_logic);
end component;
\end{verbatim}
A shared Load-Request operator.  The generic colouring is used
to implement interlock (a tag output is generated to
identify requester if relevant).  
 If suppress\_immediate\_ack(I) 
is true, then requester I will not be acked immediately even if
it is possible to do so.
}

\frame[containsverbatim]{\frametitle{Operators: load components {\bf ApLoadComplete}}
\begin{verbatim}
component ApLoadComplete
          generic(width: NaturalArray; 
                  colouring : NaturalArray);
          port(req: in BooleanArray;
               data: out StdLogicArray2D;
               ack: out BooleanArray;
               mreq: out std_logic_vector;
               mack: in std_logic_vector;
               mdata: in std_logic_vector;
               mtag : in std_logic_vector;
               clk,reset: in std_logic);
end component;
\end{verbatim}
A shared Load-Complete operator.  The generic colouring is used
to implement interlock (together with the tag input). 
}

\frame[containsverbatim]{\frametitle{Operators: store components {\bf ApStoreReq} }
\begin{verbatim}
component ApStoreReq 
          generic(width: NaturalArray; 
                  colouring: NaturalArray;
                  suppress_immediate_ack: BooleanArray);
          port(addr,data: in StdLogicArray2D;
               req: in BooleanArray;
               ack: out BooleanArray;
               mreq: out std_logic_vector;
               mack: in std_logic_vector;
               maddr,
                mdata: out std_logic_vector;
               mtag: out std_logic_vector;
               clk,reset: in std_logic);
end component;
\end{verbatim}
A shared Store-Request operator.  The generic colouring is used
to implement interlock (a tag output is generated to
identify requester if relevant). 
 If suppress\_immediate\_ack(I) 
is true, then requester I will not be acked immediately even if
it is possible to do so.
}

\frame[containsverbatim]{\frametitle{Operators: store components {\bf ApStoreComplete} }
\begin{verbatim}
component ApStoreComplete 
          generic(width: NaturalArray; 
                  colouring : NaturalArray);
          port(req: in BooleanArray;
               ack: out BooleanArray;
               mreq: out std_logic_vector;
               mack: in std_logic_vector;
               mtag : in std_logic_vector;
               clk,reset: in std_logic);
end component;
\end{verbatim}
A shared Store-Complete operator.  The generic colouring is used
to implement interlock (together with the tag input). 
}

\frame[containsverbatim]{\frametitle{Datatypes:  type {\bf ApFloat}}
The LLVM float type has two primary parameters: the length of
the characteristic and the length of the mantissa.
The overall number is signbit-characteristic-mantissa.
The equivalent VHDL type is 
\begin{verbatim}
type ApFloat is array(integer range <>)
        of std_logic;
\end{verbatim}
Thus, a signal a with a characteristic of 5 bits
and a mantissa of 3 bits will be represented as
\begin{verbatim}
signal a: ApFloat(5 downto -3);
-- a(5) is the sign,
-- a(4 downto 0) is the characteristic
-- a(-1 downto -3) is the mantissa.
\end{verbatim}
Operations on the ApFloat follow the LLVM spec wherever applicable,
and the IEEE spec where the LLVM spec is silent.
}

\frame[containsverbatim]{\frametitle{Datatypes: type {\bf ApFloatArray}}
An array of floats.
\begin{verbatim}
type ApFloatArray is 
  array(integer range <>, integer range <>) 
  of std_logic;
signal a: ApFloatArray(0 to 7, 5 downto -3);
  -- 8 ApFloats, each with
  -- characteristic of 5 bits, 
  -- mantissa of 3 bits.
\end{verbatim}
}
\frame[containsverbatim]{\frametitle{Functions: {\bf ApFloatArray} access functions }
The following extract and insert functions will be used to access
elements of the array.
\begin{verbatim}
function
   Extract (constant x: ApFloatArray;
              constant idx: integer)
   return ApFloat;
procedure
   Insert (x: out ApFloatArray;
            y: in ApFloat;
            idx: in integer);
   
\end{verbatim}
The parameter idx must be an element of \begin{verbatim} x'range(1). \end{verbatim}
}

\frame[containsverbatim]{\frametitle{Operators: {\bf ApFloat $\leftrightarrow$ ApInt} conversion operators }
\begin{verbatim}
component ApIntUnsignedToApFloat
             generic(colouring: NaturalArray);
             port (x: in ApIntArray;
                   clk,reset: in std_logic; 
                   reqR,reqC: in BooleanArray;
                   ackR,ackC: out BooleanArray;
                   z: out ApFloatArray);
end component;
\end{verbatim}
The operator  ApIntSignedToApFloat is similar to ApIntUnsignedToApFloat
}

\frame[containsverbatim]{\frametitle{Operators: {\bf ApFloat $\leftrightarrow$ ApInt} conversion operators }
\begin{verbatim}
component ApFloatToApIntUnsigned 
             generic(colouring: NaturalArray);
             port (x: in ApFloatArray; 
                   clk,reset: in std_logic; 
                   reqR,reqC: in BooleanArray;
                   ackR,ackC: out BooleanArray;
                   z: out ApIntArray);
end component;
\end{verbatim}
The operator  ApFloatToApIntSigned is similar to ApFloatToApIntUnsigned.
}
\frame[containsverbatim]{\frametitle{Operators: {\bf ApFloat} conversion operators }

\begin{verbatim}
component ApFloat_S_1 
             generic(colouring: NaturalArray);
             port (x: in ApFloatArray;
                   clk,reset: in std_logic; 
                   reqR,reqC: in BooleanArray;
                   ackR,ackC: out BooleanArray;
                   z: out ApFloatArray);
end component;
\end{verbatim}
The following configuration can be applied to
ApFloat\_S\_1.
\begin{verbatim}
ApFloatResize
\end{verbatim}
}

\frame[containsverbatim]{\frametitle{Operators: {\bf ApFloat} arithmetic operators }
\begin{verbatim}
component ApFloat_S_2
             generic(colouring: NaturalArray);
             port (x,y: in ApFloatArray; 
                   clk,reset: in std_logic; 
                   reqR,reqC: in BooleanArray;
                   ackR,ackC: out BooleanArray;
                   z: out ApFloatArray);
end component;
\end{verbatim}
The following configurations can be applied:
\begin{verbatim}
ApFloatAdd, ApFloatSub, ApFloatMul.
\end{verbatim}
}
\frame[containsverbatim]{\frametitle{Operators: {\bf ApFloat} comparison operators}
\begin{verbatim}
component ApFloat_Cmp_2
             generic(colouring: NaturalArray);
             port (x,y: in ApFloatArray; 
                   clk,reset: in std_logic; 
                   reqR,reqC: in BooleanArray;
                   ackR,ackC: out BooleanArray;
                   z: out ApIntArray);
end component;
\end{verbatim}
The following configurations can be applied:
\begin{verbatim}
ApFloatOeq, ApFloatOne, ApFloatOrd 
ApFloatUeq, ApFloatUne, ApFloatUno
ApFloatOgt, ApFloatOge, ApFloatOlt,ApFloatOle
ApFloatUgt, ApFloatUge, ApFloatUlt,ApFloatUle.
\end{verbatim}
}

\frame[containsverbatim]{\frametitle{Operators: {\bf ApFloat} arithmetic operators with 
one constant operand}
\begin{verbatim}
component ApFloat_S_2_C
             generic (y: ApFloatArray; colouring: NaturalArray);
             port (x: in ApFloatArray; 
                   clk,reset: in std_logic; 
                   reqR,reqC: in BooleanArray;
                   ackR,ackC: out BooleanArray;
                   z: out ApFloatArray);
end component;
\end{verbatim}
The following configurations can be applied to ApFloat\_S\_2\_C:
\begin{verbatim}
ApFloatAddC, ApFloatSubC, ApFloatMulC.
\end{verbatim}
}

\frame[containsverbatim]{\frametitle{Operators: {\bf ApFloat} comparison operators with 
one constant operand}
\begin{verbatim}
component ApFloat_Cmp_2_C
          generic (y: ApFloatArray; 
                   colouring: NaturalArray);
          port    (x: in ApFloatArray; 
                   clk,reset: in std_logic; 
                   reqR,reqC: in BooleanArray;
                   ackR,ackC: out BooleanArray;
                   z: out ApIntArray);
end component;
\end{verbatim}
The following configurations can be used:
\begin{verbatim}
ApFloatOeqC, ApFloatOneC, ApFloatOrdC ApFloatUeqC, 
ApFloatUneC, ApFloatUnoC ApFloatOgtC, ApFloatOgeC, 
ApFloatOltC, ApFloatOleC ApFloatUgtC, ApFloatUgeC, 
ApFloatUltC, ApFloatUleC.
\end{verbatim}
}


\frame[containsverbatim]{\frametitle{Operators: {\bf ApFloat} multiplexor operators }
\begin{verbatim}
component ApFloatPhi 
                   port(x,y: in  ApFloat; 
                        reqx,reqy: in boolean;
                        z: out ApFloat; 
                        ack: out boolean;
                        clk,reset: in bit);
end component;
component ApFloatSelect 
                   port(x,y: in  ApFloat; 
                        sel: in ApInt;
                        req: in boolean;
                        z: out ApFloat; 
                        ack: out boolean;
                        clk,reset: in bit);
end component;
\end{verbatim}
These are special, unshared operators.
}

\begin{frame}[fragile]
  \frametitle{Input Port}
\begin{alltt}
component InputPort
  generic (colouring: NaturalArray);
  port (
    -- pulse interface with the data-path
    req       : in  BooleanArray;
    ack       : out BooleanArray;
    data      : out StdLogicArray2D;
    
    -- ready/ready interface with outside world
    oreq       : out std_logic;
    oack       : in  std_logic;
    odata      : in  std_logic_vector;
    
    clk, reset : in  std_logic);
end component;
\end{alltt}
If colouring indicates that mutual 
exclusion does not hold, then {\bf one} of the requesters
will succeed.
\end{frame}

\begin{frame}[fragile]
  \frametitle{Output Port}
\begin{alltt}
component OutputPort
  generic (colouring: NaturalArray);
  port (
    req       : in  BooleanArray;
    ack       : out BooleanArray;
    data      : in  StdLogicArray2D;
    oreq       : out std_logic;
    oack       : in  std_logic;
    odata      : out std_logic_vector;
    clk, reset : in  std_logic);
end component;
\end{alltt}
If colouring indicates that mutual 
exclusion does not hold, then {\bf one} of the requesters
will succeed.
\end{frame}

\begin{frame}
  \frametitle{Function Calls}
  \begin{itemize}
  \item {\bf Call} made to another module.
    \begin{itemize}
    \item Use OutputPort.
      \begin{itemize}
      \item Data port is  an std\_logic\_vector consisting of
        arguments and address for the called module.
      \end{itemize}
    \end{itemize}
  \item {\bf Response} received for the corresponding call.
    \begin{itemize}
      \item Use InputPort.
        \begin{itemize}
        \item Return-value packed in an std\_logic\_vector.
        \end{itemize}
    \end{itemize}
  \item {\bf Accept} an incoming call.
    \begin{itemize}
    \item Use InputPort.
      \begin{itemize}
      \item Arguments packed in an std\_logic\_vector.
      \end{itemize}
    \end{itemize}
  \item {\bf Return} an incoming call.
    \begin{itemize}
    \item Use OutputPort.
      \begin{itemize}
      \item Return value packed in an std\_logic\_vector.
      \end{itemize}
    \end{itemize}
  \end{itemize}
\end{frame}

\begin{frame}[fragile]
  \frametitle{Arbiter for function calls}
\small
\begin{alltt}
entity CallArbiter is
  port ( -- ready/ready handshake on all ports
    -- ports for the caller
    call_reqs   : in  std_logic_vector;
    call_acks   : out std_logic_vector;
    call_data   : in  StdLogicArray2D;
    -- call port connected to the called module
    call_mreq   : out std_logic;
    call_mack   : in  std_logic;
    call_mdata  : out std_logic_vector;
    call_mtag   : out std_logic_vector;
    -- similarly for return, initiated by the caller
    return_reqs : in  std_logic_vector;
    return_acks : out std_logic_vector;
    return_data : out StdLogicArray2D;
    -- return from function 
    return_mreq : in std_logic;
    return_mack : out std_logic;
    return_mdata : in  std_logic_vector;
    return_mtag : in  std_logic_vector);
end CallArbiter;
\end{alltt}
\end{frame}

\end{document}


% LocalWords:  StdLogicArray ApIntArray ApFloatArray ApInt ApFloat downto ireq
% LocalWords:  InputPort iack idata oreq oack odata OutputPort addr mreq maddr
% LocalWords:  mtag CallRequest CallComplete IncomingCallRequest
% LocalWords:  IncomingCallComplete
\end{document}
